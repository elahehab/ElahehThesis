%%%%%%%%%%%%%%%%%%%%%%%%%%%%%%%%%%%%%%%%%%%%%%%
% LATEX UT THESIS TEMPLATE 							      	   	 %
% By Elahe Rannai									        	 %
% No rights reserved. The author appreciates any distribution and/or completion of this material.  	 %
%%%%%%%%%%%%%%%%%%%%%%%%%%%%%%%%%%%%%%%%%%%%%%%
\documentclass[twoside, a4paper,11pt]{book}
\usepackage[margin=10mm,font={footnotesize},labelfont={footnotesize}]{caption}

\usepackage{amsmath}
\usepackage{graphicx}
\usepackage[multidot]{grffile}
\usepackage{wrapfig}
\usepackage{verbatim}
\usepackage{fancyhdr}
\usepackage{url}
\usepackage{hyperref}
\usepackage{supertabular}
\usepackage{multicol}
\usepackage{setspace}
\usepackage{changepage}
\usepackage{color}
\usepackage{array}
\usepackage{colortbl}

\usepackage{amsmath}
\usepackage{amsthm}
\usepackage{amssymb}
\usepackage{cancel}
\usepackage{mathtools}
\usepackage{extpfeil}

\usepackage{subfigure}
\usepackage{tabularx}
\usepackage{multirow}
\usepackage{afterpage}


\usepackage{cite}
%\usepackage{cleveref}
\usepackage[rgb,x11names]{xcolor}% Optimize for screen reading.


\usepackage{algorithm}
\usepackage{algorithmic}
\usepackage[xindy]{glossaries}

\usepackage{zref-abspage}
\usepackage{perpage}
\MakePerPage{footnote}

\usepackage{hhline}
\usepackage[T1]{fontenc}
\usepackage{xcolor}    % loads also »colortbl«

\definecolor{stBlue}{RGB}{0,176,240}
\definecolor{stGreen}{RGB}{0,204,0}
\definecolor{stRed}{RGB}{255,0,0}
\definecolor{stYellow}{RGB}{255,255,0}
\definecolor{stOrange}{RGB}{255,192,0}
\definecolor{stPink}{RGB}{255,153,153}
\definecolor{stGray}{RGB}{191,191,191}
\definecolor{stPurple}{RGB}{112,48,160}
\definecolor{stBrown}{RGB}{204,153,0}
\definecolor{stLightGreen}{RGB}{153,255,153}
\definecolor{stLightPurple}{RGB}{204,153,255}



\usepackage[extrafootnotefeatures]{xepersian}
\LTRcolumnfootnotes


\numberwithin{equation}{chapter}
\numberwithin{table}{chapter}
\numberwithin{figure}{chapter}
\numberwithin{equation}{chapter}


\settextfont[Scale=1.12]{XB Niloofar}
\setlatintextfont[Scale=1.05]{Times New Roman}

\DefaultMathsDigits
\DeclareMathSizes{10}{12}{8}{6}   % For size 10 text


\defpersianfont\nastaliq[Scale=1]{IranNastaliq}
\defpersianfont\nastaliqbig[Scale=1.3]{IranNastaliq}


\linespread{2.04}
\setlength\parskip{0.25cm}
\setlength\topmargin{-0.5in}
\setlength\headheight{2cm}
\setlength\headsep{0.7cm}
\setlength\textheight{8.997in}
\setlength\textwidth{5.9078in}
\setlength\oddsidemargin{-0.0158in}
\setlength\evensidemargin{0.38in}

\setlength{\parindent}{1cm}

\newenvironment{strict_enumerate}
{\begin{enumerate}
  \setlength{\itemsep}{0.1pt}
  \setlength{\parskip}{0pt}
  \setlength{\parsep}{0pt}}
{\end{enumerate}}

\newenvironment{strict_itemize}
{\begin{itemize}
  \setlength{\itemsep}{1pt}
  \setlength{\parskip}{0pt}
  \setlength{\parsep}{0pt}}
{\end{itemize}}


\newenvironment{strict_description}
{\begin{description}
  \setlength{\itemsep}{1pt}
  \setlength{\parskip}{0pt}
  \setlength{\parsep}{0pt}}
{\end{description}}


\newglossarystyle{mylistFa}{
\glossarystyle{list}
\renewenvironment{theglossary}{}{}
\renewcommand*{\glossaryheader}{}
\renewcommand*{\glsgroupheading}[1]{}
\renewcommand*{\glsgroupskip}{}
\renewcommand*{\glossaryentryfield}[5]     {\noindent\glstarget{##1}{##2}\dotfill \space \lr{##3} \\}
\renewcommand*{\glossarysubentryfield}[6]{\glossaryentryfield{##2}{##3}{##4}{##5}{##6}}
}

\newglossarystyle{mylistLa}{
\glossarystyle{list}
\renewenvironment{theglossary}{}{}
\renewcommand*{\glossaryheader}{}
\renewcommand*{\glsgroupheading}[1]{}
\renewcommand*{\glsgroupskip}{}
\renewcommand*{\glossaryentryfield}[5]     {\noindent\glstarget{##1}{##3}\dotfill \space ##2 \\}
\renewcommand*{\glossarysubentryfield}[6]{\glossaryentryfield{##2}{##3}{##4}{##5}{##6}}
}

\newglossary[glg]{latin}{gls}{glo}{واژه‌نامه‌ انگلیسی به فارسی}
\newglossary[blg]{persian}{bls}{blo}{واژه‌نامه‌ فارسی به انگلیسی}

\newcommand{\mls}[1]{\gls{fa-#1}\glsuseri{la-#1}}

\newcommand{\inpdic}[2]{
	\newglossaryentry{fa-#1}{type=persian,name={#1}, sort={#1},description={#2}}
	\newglossaryentry{la-#1}{type=latin,name=\lr{#2}, sort={#2},description={#1}}
}


\makeatletter
\renewcommand*{\cleardoublepage}{\clearpage\if@twoside \ifodd\c@page\else
\hbox{}%
\thispagestyle{empty}%
\newpage%
\if@twocolumn\hbox{}\newpage\fi\fi\fi}

%meee error ! LaTeX Error: \gls@persian@displayfirst undefined.
%\renewcommand{\gls@persian@displayfirst}[4]{
	%#1#4\protect\LTRfootnote{#2}
%}
\defglsdisplayfirst[persian]{#1#4\protect\LTRfootnote{#2}}

%\makeatother
\makeglossaries
\glsdisablehyper
\inpdic{بازی‌های شناختی}{Cognitive Games}
\inpdic{توانمندی‌های شناختی}{Cognitive Abilities}
\inpdic{توجه و تمرکز}{Attention}
\inpdic{توجه}{Attention}
\inpdic{حافظه}{Memory}
\inpdic{حل مساله}{Problem Solving}
\inpdic{استراتژی}{Strategy}
\inpdic{استراتژی یادگیری}{Learning Strategy}
\inpdic{تکرار کردن}{Rehearsal}
\inpdic{برقرار کردن ارتباط معنایی}{Semantic}
\inpdic{گروه کردن}{Grouping}
\inpdic{تماس چشمی}{Eye contact}
\inpdic{توضیح دادن}{Paraphrasing}
\inpdic{گفتگو با خود}{self-talk}
\inpdic{توجه تقسیم شده}{Divided Attention}
\inpdic{ردیابی همزمان چندین شیء}{Multiple Object Tracking}
\inpdic{ادراک}{Perception}
\inpdic{مهارت‌های حرکتی}{Motor Skills}
\inpdic{کارکردهای اجرایی}{Executive Functions}
\inpdic{بیش‌فعالی}{ADHD}
\inpdic{اختلال کمبود توجه}{ADD}
\inpdic{فعالیت‌ها}{task}



\begin{document}
% Besmellah Page
\newpage
\thispagestyle{empty}
\begin{center}
\begin{tabular}{c}
\\ \\ \\ \\ \\
\includegraphics[scale=0.7, origin=c]{Figures/besm1.jpg}\\
\end{tabular}
\end{center}
%


\newpage
\thispagestyle{empty}
\mbox{}

%%%%%%%%%%%%%%%%%
%TITLE PAGE
%%%%%%%%%%%%%%%%%
\newpage
\thispagestyle{empty}
\begin{center}
\begin{tabular}{lp{7cm}r}
\includegraphics[width=2.8cm]{Figures/ut.png} & & \includegraphics[width=3.8cm]{Figures/eng.png} \\
\end{tabular}

{\LARGE\bfseries دانشگاه \space تهران}
\\*
{\Large\bfseries پردیس دانشکده‌های فنی}
\\*
{\Large\bfseries دانشکده‌ مهندسی برق و کامپیوتر}
\par
\vskip 1.4cm
{\Huge\bfseries ارزیابی تجربه‌ی کاربران در بازی شناختی توجه و استفاده از آنها برای بهبود عملکرد افراد مبتدی}\par
\vskip .9cm
{\large%
  نگارش }\par
{\Large\bfseries الهه ابوالحسنی شهرضا}\par
\par
{\large
  اساتید راهنما\par
\Large\bfseries دکتر مجید نیلی  \\
\Large\bfseries دکتر هادی مرادی  \\

}
\par
\vskip 1.4cm
{\large\bfseries پایان‌نامه برای دریافت درجه کارشناسی‌ارشد در رشته \\* مهندسی کامپیوتر - گرایش هوش مصنوعی}
\par
\vskip .5cm
{\large شهریور ۱۳۹۶}
\par
\vfill
\end{center}

%%%%%%%%%%%%%%%%%%%%
%\newpage
%\thispagestyle{empty}
%\mbox{}

%\newpage
%\thispagestyle{plain}
%\begin{center}
%\begin{tabular}{lp{7cm}r}
%\includegraphics[width=15cm]{Figures/Table.png} 
%\end{tabular}
%\end{center}
%%%%%%%%%%%%%%%%%%%%%%
\newpage
\thispagestyle{empty}
\mbox{}

\newpage
\thispagestyle{plain}
\begin{center}
\begin{tabular}{lp{7cm}r}
\includegraphics[width=2.8cm]{Figures/ut.png} & & \includegraphics[width=3.8cm]{Figures/eng.png} \\
\end{tabular}

{\LARGE\bfseries دانشگاه \space تهران}
\\*
{\Large\bfseries پردیس دانشکده‌های فنی}
\\*
{\Large\bfseries دانشکده مهندسی برق و کامپیوتر}
\par
پایان نامه برای دریافت درجه کارشناسی ارشد در رشته مهندسی کامپیوتر

\vskip 1cm
\par
عنوان:

{\Large\bfseries ارزیابی تجربه‌ی کاربران در بازی شناختی توجه و استفاده از آنها برای بهبود عملکرد افراد مبتدی}

\vskip .2cm
{\Large نگارش: الهه ابوالحسنی شهرضا}
\end{center}
\vskip .3cm

\noindent
این پایان‌نامه در تاریخ ۱۳۹۶/۰۶/۱۲ در مقابل هیأت داوران دفاع گردید و مورد تصویب قرار گرفت.
\vskip .5cm
\begin{tabular}{r r}
معاون آموزشی و تحصیلات تکمیلی پردیس دانشکده‌های فنی: دکتر جلیل آقا راشد محصل \\
رئیس دانشکده مهندسی برق و کامپیوتر: دکتر مجید نیلی\\
معاون پژوهشی و تحصیلات تکمیلی: دکتر بابک نجار اعرابی \\
استاد راهنما: آقای دکتر مجید نیلی \\
استاد راهنمای دوم: آقای دکتر هادی مرادی\\
عضو هیأت داوران:  \\
عضو هیأت داوران:  \\

\end{tabular}



\newpage
\thispagestyle{empty}
\mbox{}

\newpage
\thispagestyle{empty}
{ \LARGE 
\begin{center}
تعهدنامه اصالت اثر
\end{center}}
اینجانب الهه ابوالحسنی شهرضا تایید می‌کنم که مطالب مندرج در این پایان نامه حاصل کار و پژوهش اینجانب بوده و به دستارود‌های پژوهشی دیگران که در این نوشته از آنها استفاده شده است مطابق مقررات ارجاع گردیده است. به‌علاوه این پایان نامه قبلا برای احراز هیچ مدرک هم سطح یا بالاتر ارائه نشده است.
\par
کلیه حقوق مادی و معنوی این اثر متعلق به دانشکده فنی دانشگاه تهران است.
\par 
\vskip 2cm


\begin{flushleft}
نام و نام خانوادگی دانشجو: الهه ابوالحسنی شهرضا \hspace{1.75cm}~~~~~~~ \\
امضای دانشجو: ~~~~~~~~~~~~~~~~~~~~~~~~~~~~~~~~~~~~~~~~~~~~~~~~~~~~~
\end{flushleft}

\newpage
\thispagestyle{empty}
\mbox{}


%\newpage
%\thispagestyle{empty}
%\mbox{}
%\includegraphics[scale=0.9]{Figures/taghdir.pdf} 


\newpage
\thispagestyle{empty}
\mbox{}


\newpage
\thispagestyle{empty}
{\nastaliqbig \Huge 
تقدیر و تشکر\nastaliq

{\par\vspace{1cm}}
\LARGE
\begin{adjustwidth}{1cm}{}
مراتب تشکر و قدردانی خود را نسبت به تمام کسانی که مرا در انجام این پایان‌نامه یاری کرده‌اند، خصوصاً اساتید گرامی و ارجمندم، جناب آقای دکتر نیلی و جناب آقای دکتر مرادی که رهنمودهای ایشان همواره راهگشای پیچیدگی‌های این پژوهش بوده است، ابراز می‌دارم. 

همچنین، از تمامی دوستانم در آزمایشگاه رباتیک شناختی که با حضور و محبتشان مرا یاری نمودند تشکر و قدردانی می‌نمایم. 

\end{adjustwidth}
}


\newpage
\thispagestyle{empty}
\mbox{}


\pagestyle{plain}

\newpage


\vspace*{-2.6cm}
\subsection*{چکیده}

\vspace*{-.4cm}

\begin{spacing}{1.5}

در سال‌های اخیر گروه‌های زیادی اقدام به طراحی و توسعه‌ی بازی‌های شناختی کامپیوتری کرده‌اند که هدف آن کمک به ارتقاء توانمندی‌های شناختی مانند توجه و تمرکز، حافظه و حل مساله است. لیکن نتیجه‌ی مطالعات بر روی اثربخشی این بازی‌ها در ارتقاء قابلیت‌های شناختی متناقض است. این تناقض در پاسخ به این سوال است که آیا راهبردهای خودیافته که افراد جهت بهبود عملکردهای شناختی خود در محیط بازی از آن استفاده می‌کنند، قابل تعمیم به دنیای واقعی نیز است یا به محیط بازی منحصر می‌شود. 

با در نظر گرفتن این موضوع که استفاده‌ی آگاهانه از یک راهبرد، احتمال استفاده از آن را در دنیای واقعی بیشتر می‌کند در این پژوهش بر آنیم که به دو سوال اساسی پاسخ دهیم. نخست آن که  آیا تفاوت بین عملکرد افراد، در اثر استفاده‌ی آنها از راهبردهای متفاوت ایجاد می‌شود یا ناشی از تفاوت در ویژگی‌های فردی آنهاست. دیگر آن که در صورت یافتن یک یا چند راهبرد موثر، آیا این راهبرد قابلیت آموزش دارد و منجر به اثربخشی بیشتر در بازی می‌شود یا خیر. در ذیل سوال نخست به دو موضوع پرداخته می‌شود. موضوع اول جمع‌آوری راهبردهای مورد استفاده افراد و موضوع دوم استخراج راهبردهای موثر است. برای این کار از یک بازی شناختی توسعه داده شده توسط مجموعه‌ی مغزینه در حوزه‌ی «توجه تقسیم شده» استفاده شد که ویژگی‌ها و مراحل آن متناسب با نیازهای پژوهش بازطراحی شدند. از افراد خواسته شد در مدت زمان محدود بازی نموده و راهبردهای مورد استفاده‌ی خود را با استفاده از پرسشنامه گزارش کنند. آزمون تا جایی ادامه پیدا می‌کند که راهبرد جدیدی گزارش نشود. برای استخراج راهبردها از پرسشنامه‌ها، از روش تحلیل محتوا و کدگذاری پاسخ‌ها استفاده شد. سپس با استفاده از روش‌های آماری نمره‌ی هر کدام از راهبردها محاسبه شد. در نهایت راهبردهای به دست آمده به سه گروه اصلی تقسیم‌بندی شدند و مشخص شد یکی از این سه گروه نسبت به دو گروه دیگر اثربخشی بیشتری دارد. در ادامه برای پاسخ به سوال اصلی دوم، دو موضوع بررسی شدند. اول اینکه ارتباط میان میزان یادگیری راهبرد و بهبود عملکرد افراد به چه صورت است و دوم اینکه این میزان چه ارتباطی با مدل محور یا مدل آزاد بودن روش یادگیری فرد دارد. برای پاسخ دادن به این سوالات یک آزمون سه مرحله‌ای طراحی شد که در  مرحله اول میزان مدل محور بودن یادگیری فرد معلوم می‌شود. در دو مرحله‌ی بعدی از بازی بخش اول استفاده می‌شود. در  مرحله دوم، بازی انجام شده و راهبردها توسط افراد گزارش می‌شوند. سپس راهبردهای موثرتر به دو صورت کلامی و با استفاده از راهنمای داخل بازی آموزش داده می‌شود. در نهایت در مرحله سوم بازی مجدداً تکرار شده و از فرد خواسته می‌شود میزان استفاده از راهبرد آموزش داده شده را گزارش کند. در کنار این آزمون یک آزمون دو مرحله‌ای از گروه کنترل گرفته می‌شود که فاقد مرحله‌ی اول و آموزش راهبرد است. این مراحل بدون وقفه و پشت سر هم انجام می‌شوند. 

در نتایج به دست آمده مشاهده می‌کنیم ارتباط معناداری بین مدل محور بودن فرد و میزان یادگیری راهبرد وجود ندارد. همچنین بهبود عملکرد افرادی که راهبرد به آنها آموزش داده نشده به طرز معناداری بیشتر از افرادی است که راهبرد به آنها آموزش داده شده است.

\textbf{واژه‌های کلیدی: }\textit{راهبرد، بازی شناختی، توجه و تمرکز، توجه تقسیم‌شده، مدل محور}
\end{spacing}

\pagenumbering{harfi}


\newpage
\thispagestyle{empty}
\mbox{}

\renewcommand\listfigurename{فهرست شکل‌ها}
\renewcommand\listtablename{فهرست جدول‌ها}
%\renewcommand{\refname}{مراجع}

\begin{doublespace}\small{
\tableofcontents
\listoftables
\listoffigures
}\end{doublespace}





\pagestyle{fancy}	
\fancyhead{} 
\fancyhead[RO]{\leftmark}
\fancyhead[LO]{\thepage}
\fancyhead[LE]{\rightmark}
\fancyhead[RE]{\thepage}
\fancyfoot{} 
\renewcommand{\headrulewidth}{0.6pt} 
\renewcommand{\footrulewidth}{0pt}

\chapter{مقدمه}
\label{chapter:introduction}
\pagenumbering{arabic}\setcounter{page}{1}
\thispagestyle{plain}

طبق تعریف \mls{بازی‌های شناختی} بازی‌هایی هستند که هدف آنها بهبود \mls{توانمندی‌های شناختی} بازیکنان است. در این بازی‌ها سعی می‌شود توانمندی‌های شناختی مانند \mls{توجه و تمرکز}، \mls{حافظه}  و \mls{حل مساله}  بهبود پیدا کنند. این بازی‌ها به منظور استفاده‌ی عموم مردم طراحی شده‌اند. با وجود توسعه و استفاده‌ی روز افزون از این بازی‌ها نتایج برخی از تحقیقات انجام شده (\cite{melby2013WM}، \cite{redick2013Intellig}) نشان می‌دهند در بسیاری از موارد تاثیرگذاری مورد انتظار را نداشته‌اند.\\
هدف نهایی بازی‌های شناختی بهبود توانمندی‌های شناختی افراد است. بهبود این توانمندی‌ها در هر فرد باعث بهبود کیفیت زندگی او می‌شود و آسیب‌های شناختی احتمالی ناشی از کهولت سن یا حوادث را به تعویق می‌اندازد. در اصل این بازی‌ها نوعی ورزش مغزی محسوب می‌شوند.\\
پرسشی که ایجاد می‌شود این است که آیا می‌توان با تغییر دادن بازی‌های شناختی و شخصی سازی آنها به بهبود تاثیرگذاری این بازی‌ها کمک کرد؟ آیا می‌توان با توجه به نقاط ضعف و قوت بازیکن به نحوی بازی را تغییر داد که بیشترین تاثیرگذاری ممکن اتفاق بیافتد؟\\
یکی از روش‌های رایج به منظور بهبود عملکرد افراد در حوزه‌های مختلف، مانند توانبخشی شناختی []، آموزش زبان دوم [] و یا عملکرد دانشگاهی [] آموزش \mls{راهبرد} است. منظور از راهبرد در این پژوهش، \mls{راهبرد یادگیری}  است. راهبرد یادگیری به فرآیندهایی گفته می‌شود که وقتی با نیازمندی‌های یک تمرین مطابقت پیدا می‌کنند باعث بهبود عملکرد می‌شوند \cite{donker2014LearningSt}.\\
راهبردهای مربوط به بهبود حافظه شناخته‌شده‌تر هستند. به عنوان مثال می‌توان به استراتژي \mls{تکرار کردن} ، \mls{برقرار کردن ارتباط معنایی}  یا \mls{گروه کردن}  اشاره کرد \cite{morrison2016WM}. اما روی راهبردهای مربوط به بهبود توجه کمتر کار شده است. از راهبرد‌های شناخته شده در حوزه‌ی توجه می‌توان به \mls{تماس چشمی} ، \mls{توضیح دادن}  یا \mls{گفتگو با خود} اشاره کرد \cite{twamley2008CogTrain}.\\
در این پژوهش هدف نهایی طراحی یک بازی شناختی تطبیق‌پذیر با بازیکن است که می‌تواند با توجه به نحوه‌ی عملکرد او روش بازی‌اش را استخراج کند و سپس راهبرد‌هایی را به او آموزش دهد که باعث بهبود عملکرد وی در بازی و نهایتا در زندگی واقعی می‌شود.\\
راهبرد‌های مربوط به توجه تا به امروز مورد توجه قرار نگرفته بودند و راهبرد‌های بسیار محدودی برای آن مطرح شده بود. در این پژوهش سعی شده است مجموعه‌ای از راهبرد‌های مربوط به «\mls{توجه تقسیم شده}» معرفی شوند.\\
علاوه بر این، این پژوهش یک چارچوب جهت استفاده از آموزش راهبرد در بازی‌های مختلف ارائه می‌دهد که می‌توان از آن برای بازی‌های دیگر نیز استفاده کرد.\\
به منظور دستیابی به اهداف این پژوهش یک بازی شناختی با نام «ابر باران‌زا» انتخاب شد که هدف اصلی آن بهبود شاخه‌ی «توجه تقسیم‌شده» از زیرشاخه‌های «توجه» است.\\
این پژوهش شامل دو مرحله اصلی است. مرحله اول را مرحله استخراج راهبرد و مرحله دوم را مرحله انتقال راهبرد می‌نامیم.\\
در مرحله اول دو هدف پیگیری می‌شوند. هدف اول گردآوری راهبرد‌هایی است که افراد در این بازی استفاده می‌کنند و هدف دوم بررسی میزان اثرگذاری این راهبرد‌ها است. به این معنا که هر کدام از این راهبرد‌ها به صورت میانگین چقدر توانسته‌اند برای این بازیکن امتیاز به دست بیاورند.\\
در مرحله دوم هدف بررسی تاثیر انتقال این راهبرد‌ها به افرادی است که عملکرد ضعیف‌تری داشته‌اند. (تکمیل شود)\\







%%%%%%%%%%%%%%%%%%%%%%%%%%%%%%%%%%%%%%%%%%%%%%%%%%%%%%
%\newpage
%\thispagestyle{empty}
%\mbox{}

\chapter{پژوهش‌های پیشین}
\label{chapter:relatedWork}
\thispagestyle{plain}
در این فصل، پژوهش‌های پیشین را در سه بخش ارائه می‌دهیم: بازی‌های شناختی، استفاده از آموزش راهبرد در بازی‌ها، پژوهش‌های انجام شده در حوزه‌ی فعالیت \mls{ردیابی همزمان چندین شیء}\\
\section{بازی‌ها و تمرین‌های شناختی}
بازی‌های شناختی بازی‌هایی هستند که تلاش می‌کنند توانمندی‌های شناختی افراد را تقویت کنند. توانمندی‌های شناختی مهارت‌های ذهنی هستند که برای انجام دادن ساده‌ترین تا پیچیده‌ترین کارها مورد نیاز هستند. این مهارت‌ها شامل \mls{ادراک}، \mls{توجه} ، \mls{حافظه} ، \mls{مهارت‌های حرکتی} و\mls{کارکردهای اجرایی} هستند.
یکی از مهم‌ترین مسائلی که در رابطه با بازی‌های شناختی مطرح می‌شود مساله‌ی میزان تاثیرگذاری این بازی‌ها است. سوالاتی که در این زمینه مطرح می‌شوند از این دست هستند: آیا بازی کردن با یک بازی مخصوص حافظه باعث می‌شود حافظه‌ی فرد بازیکن بهبود پیدا کند؟ چه مدت باید این بازی صورت بگیرد؟ بازیکن باید چه شرایطی داشته باشد؟ مدت‌زمان تاثیرگذاری بازی چه مدت است؟

یکی دیگر از سوالاتی که مطرح می‌شود این است که آیا تمرین کردن یک تمرین که روی یک توانایی شناختی تمرکز دارد باعث بهبود سایر توانمندی‌های شناختی نیز می‌شود؟ به این پدیده به اصطلاح \mls{انتقال} گفته می‌شود. به عنوان مثال آیا انجام دادن تمرین در حوزه‌ی \mls{حافظه‌ی کاری} باعث بهبود \mls{هوش سیال} هوش سیال می‌شود؟

در سال ۲۰۱۱ جائگی  \cite{jaeggi2011shortLong} تاثیرگذاری طولانی مدت و کوتاه مدت تمرین‌های شناختی را بررسی کرد. همچنین در تاثیرگذاری تمرین در یک حوزه‌ی شناختی بر بهبود عملکرد در یک حوزه‌ی شناختی دیگر بررسی شده است و نتیجه گرفته شده است که این دو حوزه بر یکدیگر اثرگذار هستند. به این نوع اثرگذاری در ادبیات این پژوهش انتقال  گفته می‌شود. در  \cite{jaeggi2011shortLong} انجام دادن تمرین‌هایی مرتبط با حوزه‌ی حافظه‌ی کاری انجام گرفته است و در نهایت عملکرد افراد در تمرین‌های مرتبط با هوش سیال  ارزیابی شده است. نتیجه‌ی نهایی این است که افرادی که تمرین‌های مربوط به حافظه‌ی کاری را انجام داده‌اند در تمرین‌های مربوط به هوش سیال بهتر عمل کرده‌اند و در نتیجه انتقال اتفاق افتاده است. \cite{jaeggi2011shortLong}  همچنین به بررسی تاثیر طولانی (۲ ماه) مدت این انتقال پرداخته است و نتیجه گرفته است که این تاثیرات در طولانی مدت نیز وجود داشته‌اند. 

در سال ۲۰۱۳ ملبی لرواگ \cite{melby2013WM} در یک پژوهش جامع مقالات متعددی را که تاثیرگذاری تمرین‌های شناختی در حوزه‌ی  \mls{حافظه‌ی کاری} را بررسی کرده بودند ارزیابی کرد. در این مقاله ابتدا معیارهایی برای سنجش یک پژوهش صحیح در این حوزه معرفی شده‌اند و سپس پژوهش‌های متعددی از نظر تاثیرگذاری ارزیابی شده‌اند. ملبی لرواگ در این پژوهش معتقد است یکی از دلایل تشتت آرا در زمینه‌ی تاثیرگذاری تمرین‌های شناختی استاندارد نبودن پژوهش‌ها و روش‌هایی است که در آنها استفاده شده است. نهایتا با توجه به معیارهای معرفی شده ۲۳ مطالعه‌ی انجام شده بررسی شده‌اند. در نهایت نتیجه‌ای که از این پژوهش گرفته شده این است که تمرینات شناختی در حوزه‌ی حافظه‌ی کاری باعث می‌شوند فرد در کوتاه مدت و در تمرینات مشابه در همان زمینه عملکرد بهتری داشته باشد اما شواهد کافی برای برای اثبات تاثیرگذاری بر سایر حوزه‌ها وجود ندارد.

همچنین ردیک \cite{redick2013Intellig} در سال ۲۰۱۳ در یک پژوهش تاثیر تمرین‌های مرتبط با حافظه‌ی کاری را روی چندین حوزه‌ی مختلف، مانند \mls{هوش سیال}، \mls{انجام چند کار همزمان}، \mls{ظرفیت حافظه‌ی کاری} و \mls{هوش متبلور} بررسی کرد. در این این پژوهش گروهی از نوجوانان طی ۲۰ جلسه تمریناتی را انجام دادند. آنها قبل از شروع دوره، در میانه‌ی آن و پس از اتمام آن آزمون‌هایی را انجام دادند تا روند پیشرفتشان بررسی شود. علاوه بر افراد اصلی دو گروه کنترلی نیز در مطالعه شرکت داشتند. یک گروه یک تمرین جانبی را در این مدت انجام می‌دادند و گروه دیگر هیچ تمرینی را انجام ندادند. نتایج نشان می‌دهد با وجود اینکه افراد در تمرین‌های انجام شده پیشرفت خوبی داشتند ولی در سایر حوزه‌های شناختی هیچ بهبودی نداشتند.

در سال ۲۰۱۴ جائگی نقش تفاوت‌های فردی را در تاثیرپذیری از تمرین‌های شناختی و میزان انتقال بررسی کرد \cite{jaeggi2013IndiDiff}. ادعایی که مطرح می‌کند این است که دلیل متغیر بودن نتایج مربوط به تحقیقات حوزه‌ی تمرین‌های شناختی می‌تواند تفاوت‌های فردی بین شرکت‌کنندگان باشد که در نظر گرفته نشده است. در این پژوهش جائگی نقش \mls{انگیزه} را به عنوان یک تفاوت فردی بررسی می‌کند. به همین منظور و برای اینکه انگیزه‌ی غیر واقعی ایجاد نکنند از پرداخت پول به شرکت‌کنندگان آزمون خودداری کردند. دو معیار ارزیابی برای انگیزه در نظر گرفته شده است. اولین معیار در رابطه با میزان لذتی است که فرد به علت سختی آزمون تجربه می‌کند و دومین معیار درباره‌ی باور فرد درمورد هوش است. تفاوت بین افرادی که هوش را ثابت می‌پندارند و باور به تغییرپذیری آن ندارند و افرادی که باور دارند هوش تغییر پذیر است بررسی شده است. در نهایت نتیجه گرفته شده است باور فرد درباره‌ی هوش روی میزان \mls{انتقال} تاثیرگذار است. علاوه بر این شرکت‌کنندگان در این پژوهش نسبت به سایر پژوهش‌هایی که به آنها پول پرداخت شده بود از نتایج بهتری برخوردار بودند. اما معیار اول تاثیری روی نتایج شرکت‌کنندگان و میزان انتقال نداشت.

در سال ۲۰۱۵ شات تاثیر یک بازی شناختی و یک بازی ویدئویی روی توانمندی‌های شناختی را با هم مقایسه کرد. \cite{shute2015lumoPortal2} قبل و بعد از انجام مداخله دو تست انجام شده است که در آن توانایی \mls{حل مساله}، \mls{مقاومت در برابر چالش} و \mls{تجسم فضایی} ارزیابی شده است. نتایج این بررسی نشان می‌دهد بهبود نتایج افرادی که بازی ویدئویی را انجام داده‌اند از نظر آماری معنادار است اما در بازی شناختی (که در این پژوهش از مجموعه‌ی لوموسیتی استفاده شده است) از نظر آماری معنادار نیست.

در سال ۲۰۱۵ پژوهش جامعی به بررسی تاثیر تمرین حافظه‌ی کاری بر هوش سیال پرداخت \cite{sheehan2015FluidMeta}. در این پژوهش آو و شیهان مجموعا ۲۰ تحقیق انجام شده در این حوزه را بررسی کرده‌اند که در آنها افراد سالم مورد بررسی قرار گرفته‌اند که بازه‌ی سنی آنها بین ۱۸ تا ۵۰ سال است و حتما از گروه کنترلی استفاده شده است. در نهایت به این نتیجه رسیده‌اند تمرین حافظه‌ی کاری باعث بهبود اندکی در هوش سیال می‌شود که از نظر آماری معنادار است.

در سال ۲۰۱۶ ملبی لرواگ پژوهشی مشابه کار انجام شده در سال ۲۰۱۳ انجام داد و مجموعه‌ای از تحقیقات انجام شده روی حوزه‌ی حافظه‌ی کاری را بررسی کرد \cite{melby2016WMNoImprove}. در این پژوهش جامع ملبی لرواگ قصد داشت میزان انتقال در طولانی مدت را بررسی کند. در نهایت نتیجه‌ای که این پژوهش در بر داشته، این است که تمرینات مربوط به حافظه‌ی کاری تاثیرات کوتاه مدت و مشخصی دارند ولی این تاثیرات باعث بهبود توانایی‌های شناختی در دنیای واقعی نمی‌شوند.

همانطور که نشان داده شد بررسی‌های مختلف نتایج متفاوتی در این زمینه را نشان می‌دهند. در این پژوهش هدف ما اضافه کردن آموزش راهبرد به بازی و بررسی تاثیر آن بر عملکرد بازیکن است.

\section{استفاده از آموزش راهبرد در بازی‌ها}
یکی از روش‌های کمک به افراد در راستای بهبود عملکرد آنها \mls{آموزش راهبرد} است. از این روش در آموزش موضوعات مختلف مانند آموزش آکادمیک و به ویژه آموزش زبان دوم استفاده می‌شود. ( \cite{oxford1989LangStrategy}، \cite{nunan2017Lan}، \cite{donker2014Academic}) همچنین بسیاری از تحقیقات انجام شده در حوزه‌ی آموزش شناختی به خصوص درحوزه‌ی حافظه کاری از این روش استفاده کرده‌اند. در سال ۲۰۰۱ مک‌نامارا \cite{mcnamara2001MemCap} از آموزش راهبرد برای بهبود عملکرد افراد در یک فعالیت مربوط به حافظه کاری کمک گرفت. او افراد را به دو دسته تقسیم‌بندی کرد. به یک دسته راهبرد قصه‌سازی را یاد داد تا از آن برای انجام فعالیت استفاده کنند و به دسته‌ی دیگر هیچ راهبرد آموزش نداد و از آنها خواست تا فعالیت را انجام دهند. در نهایت گروه اول بهبود چشمگیری در مقایسه با گروه دوم داشتند که این موضوع نشان‌دهنده‌ی تاثیرگذاری آموزش راهبرد در این آزمایش بوده است.

در سال ۲۰۰۷ کرتی دو گروه از بزرگسالان جوان و پیر را انتخاب کرده و اثر آموزش راهبرد روی آنها را بررسی کرد \cite{carretti2007StaMem}. فعالیتی که در نظر گرفته بود یک فعالیت مرتبط با حوزه‌ی حافظه‌ی کاری بود. افراد به دو گروه دسته‌بندی شدند و به یک گروه راهبرد آموزش داده می‌شد و گروه دیگر بدون آموزش راهبرد فعالیت را انجام می‌دادند. در نهایت نتیجه نشان داد افرادی که از راهبرد استفاده کرده بودند، چه در گروه بزرگسالان پیر چه در گروه بزرگسالان جوان عملکرد بهتری داشتند که از نظر آماری معنادار بوده است.

در سال ۲۰۱۶ موریسون توزیع راهبرد‌های مختلف بر اساس ویژگی‌های تسک‌های حافظه‌ی کاری را استخراج کرد \cite{morrison2016variation}. به این منظور تسک‌های مختلف جمع‌آوری شدند، ویژگی‌های آنها استخراج شد و در نهایت رابطه‌ی این ویژگی‌ها با راهبرد‌های استفاده شده در این تسک‌ها بررسی شد. نتیجه‌ی به دست آمده ارتباط بین استفاده از راهبرد‌ها و ویژگی‌های تسک بود. علاوه بر این نشان داده شد که افراد از یک راهبرد ثابت در تمام تسک‌ها استفاده نمی‌کنند و در نهایت کاربرها را بر اساس راهبرد مورد استفاده‌ی آنها دسته‌بندی شدند.

در پژوهشی دیگر در سال ۲۰۱۶ پنگ تاثیر انجام دادن یک فعالیت در حوزه‌ی \mls{حافظه‌ی کاری کلامی} با استفاده از آموزش راهبرد و بدون استفاده از آموزش راهبرد روی بهبود حافظه‌ی کاری کلامی و \mls{درک مطلب شنیداری} را بررسی کرد \cite{peng2016WMTrain}. کودکان کلاس اول که در معرض مشکلات یادگیری قرار داشتند هدف این آزمایش بودند. نتیجه‌ای که در نهایت به دست آمد این بود که عملکرد گروه‌هایی که درگیر تمرین بودند (چه با آموزش راهبرد و چه بدون آن) در مقایسه با گروه کنترل بهبود پیدا کرده بود. عملکرد گروهی که راهبرد به آنها آموزش داده شده بود بهتر از عملکرد گروهی بود که راهبرد را دریافت نکرده بودند ولی این مقدار از لحاظ آماری معنادار نبوده است. 

تمام پژوهش‌های بررسی شده در این بخش روی حوزه‌ی حافظه‌ی کاری تمرکز کرده بودند و همانطور که مشاهده شد تاثیر آموزش راهبرد در تمام آنها مثبت بوده است. در این پژوهش قصد داریم تاثیر آموزش راهبرد را روی یک بازی مرتبط با حوزه‌ی توجه بسنجیم.



\section{پژوهش‌های انجام شده در حوزه‌ی فعالیت ردیابی همزمان چند شیء}
فعالیت انتخاب شده در این پژوهش با استفاده از فعالیت مشابهی به نام ردیابی همزمان چند شیء طراحی شده است. در این فعالیت ابتدا تعدادی شکل به عنوان هدف و تعدادی شکل به عنوان \mls{منحرف کننده} به صورت ثابت در صفحه قرار دارند. رنگ یا شکل هدف‌ها با منحرف کننده‌ها متفاوت است. سپس با آغاز فعالیت تمامی شکل‌ها شروع به حرکت می‌کنند و هدف‌ها کم کم شبیه منحرف‌کننده‌ها می‌شوند تا زمانی که کاملا به یک شکل در بیایند. معمولا روند فعالیت به این صورت است که پس از ایستادن تمامی شکل‌ها کاربر باید هدف‌ها را از بین بقیه انتخاب کند.

فهد در پژوهشی که در سال ۲۰۰۸ انجام داد محل نگاه افراد به هنگام انجام دادن این فعالیت را بررسی کرد \cite{fehd2008whereLook}. او ۲ حالت مختلف را بررسی کرد که در آنها ۱ یا ۳ هدف در صفحه قرار داشتند. سپس بررسی کرده است که در حین انجام این فعالیت کاربران به کدام بخش صفحه بیشتر نگاه می‌کنند. در حالتی که ۳ هدف در صفحه وجود داشته افراد ۳ نوع رفتار مختلف از خود بروز داده‌اند. دسته‌ی اول به یک نقطه در میان صفحه خیره شده‌اند، دسته‌ی دوم حرکت کلی هدف‌ها را دنبال کرده‌اند و دسته‌ی سوم به تمام هدف‌ها نگاه کرده‌اند به این صورت که نقطه‌ی تمرکز چشم‌شان با سرعت بالا بین آنها جابجا شده است. در نهایت فهد نتیجه گرفته است که افراد بیشتر به مرکز شکل‌ها نگاه می‌کنند و تحلیلی که ارائه داده این است که آنها چند شکل را به صورت یک شیء در نظر می‌گیرند و آن را دنبال می‌کنند.

فهد ادامه‌ی کار قبلی را در پژوهشی در سال ۲۰۱۰ دنبال کرد \cite{fehd2010centerLook}. در این پژوهش این موضوع را بررسی می‌کند که چه عواملی روی استفاده از این راهبرد تاثیر می‌گذارند و این راهبرد چقدر برای دنبال کردن چندین هدف موثر است. در واقع در این پژوهش فهد تلاش کرده دلیل نگاه کردن به مرکز اشیاء توسط افراد را متوجه شود. به این منظور فرضیه‌های مختلف را مد نظر قرار داده و سه آزمایش انجام داده است. در آزمایش اول این موضوع را بررسی کرده است که آیا افراد به خاطر کاهش \mls{حرکات چشم} از روش نگاه کردن به مرکز استفاده می‌کنند که در نهایت به این نتیجه رسیده که اینطور نیست. آزمایش دوم بررسی کرده است که آیا افراد به این دلیل این کار را می‌کنند که با نگاه کردن به مرکز، اطلاعات مربوط به بخش‌های جانبی را از دست نمی‌دهند و در آزمایش سوم عملکرد افرادی را که از راهبرد نگاه به مرکز استفاده کرده‌اند با افرادی که هدف‌ها را مستقیما دنبال کرده‌اند مقایسه کرده است. نتیجه‌ای که از این بخش گرفته شده است این است که افرادی که از این راهبرد استفاده کرده‌اند در نهایت عملکرد بهتری داشته‌اند. 

\section{یادگیری مبتنی بر مدل و غیرمبتنی بر مدل}

یکی از زیرشاخه‌های \mls{یادگیری ماشین}، \mls{یادگیری تقویتی} است. در این روش عامل سعی می‌کند طوری در محیط عمل کند که بیشترین پاداش را دریافت کند. برای تعریف کردن یک مساله‌ی یادگیری تقویتی نیاز داریم که موارد گفته شده در رابطه‌ی \ref{eq:RLParams} را تعریف کنیم.

\begin{equation}
\label{eq:RLParams}
\begin{cases}
	S: \text{\lr{a set of states}} \\
	A:  \text{\lr{a set of actions}} \\
	P_a(s, s') = p(s_{t+1} = s'| s_t=s, a_t=a):  \text{\lr{probability of transition from state s to state s' under action a}} \\
	R_a(s, s'):  \text{\lr{expected immediate reward after transition from s to s' with action a}}

\end{cases}
\end{equation}

اگر تمامی اطلاعات گفته شده در رابطه‌ی \ref{eq:RLParams} موجود باشد مساله به راحتی قابل حل است. اما در بسیاری از موارد بخشی از اطلاعات موجود نیست. یکی از مواردی که ممکن است در ابتدای حل مساله در دست نباشد $P_a(s, s')$ و $R_a(s, s')$ است که مدل محیط را مشخص می‌کنند. بر اساس این مدل وقتی فردی در یک محیط قرار می‌گیرد که با ویژگی‌های آن آشنا نیست می‌تواند با دو رویکرد متفاوت روند یادگیری را طی کند. یک رویکرد رویکرد مدل محور است. در این رویکرد فرد تلاش می‌کند مدل محیط را بیاموزد و با استفاده از آن عمل‌هایی را انجام دهد که بهترین نتیجه را داشته باشند. رویکرد دیگر رویکرد مدل آزاد است. در این رویکرد برخلاف رویکرد قبلی فرد تلاشی برای یادگیری مدل محیط نمی‌کند و با استفاده از روش‌های دیگر سعی می‌کند بیشترین پاداش را کسب کند.

در سال ۲۰۱۱ دا در پژوهشی که انجام داد \cite{daw2011ModelBased} آزمایشی طراحی کرد که با استفاده از آن می‌توان فهمید افراد در حین یادگیری چه مقدار مدل محور و چه مقدار مدل آزاد رفتار می‌کنند. در این پژوهش برای بررسی میزان مدل محور یا مدل آزاد بودن افراد از این آزمایش که با نام \mls{آزمون دو مرحله‌ای دا} شناخته می‌شود استفاده خواهیم کرد.

\section{نتیجه گیری}

در این فصل مطالعات انجام شده در حوزه‌ی بازی‌های شناختی، استفاده از راهبرد و فعالیت ردیابی همزمان چند شیء و در نهایت مدل یادگیری مدل محور و مدل آزاد را مرور کردیم. مشاهده کردیم که پژوهش‌های بسیاری در سال‌های اخیر بازی‌های شناختی و میزان اثربخشی آنها را مورد بررسی قرار داده‌اند. در فصل‌های آتی نشان خواهیم داد چگونه با استفاده از استخراج و انتقال راهبرد تلاش کرده‌ایم اثربخشی یک بازی شناختی مشابه فعالیت ردیابی همزمان چند شیء را بهبود بخشیم.



%%%%%%%%%%%%%%%%%%%%%%%%%%%%%%%%%%%%%%%%%%%%%%%%%%%%%%
%\newpage
%\thispagestyle{empty}
%\mbox{}
\chapter{آزمایش اول - استخراج راهبرد}
\label{chapter:partOneExtraction}
\thispagestyle{plain}

\section{مقدمه}
هدف از انجام این پژوهش بررسی تاثیرگذاری آموزش راهبرد بر کارآیی کاربر در یک بازی شناختی در حوزه‌ی توجه و تمرکز است. در حوزه‌های دیگر مانند حافظه کارهای مشابه صورت گرفته است (\cite{morrison2016WM} , \cite{melby2013WM}) بنابراین یکی از دلایل انتخاب حوزه‌ی توجه و تمرکز شناخته شده نبودن راهبرد‌های مطرح در این حوزه بود. راهبرد‌های حوزه‌ی حافظه به قدری شناخته شده هستند که بسیاری از ما هنگام امتحانات مدرسه از آنها استفاده کرده‌ایم. یعنی به صورت عمومی افراد جامعه از آنها استفاده می‌کنند. اما در مورد توجه و تمرکز با اینکه تقاضا زیاد است اما راهبرد‌های مرتبط با آن به اندازه‌ی حافظه شناخته شده نیست.\\
علاوه بر این این حوزه به خودی خود از اهمیت بالایی برخوردار است. بسیاری از افراد از نداشتن تمرکز به هنگام انجام کارهای روزمره‌ی خود شکایت دارند. همچنین اختلالات زیادی در حوزه‌ی توجه و تمرکز وجود دارند (مانند \mls{اختلال کمبود توجه} یا \mls{بیش‌فعالی}).\\
شاید در این نقطه بگوییم بسیاری از حوزه‌های شناختی مانند توجه هستند که هم راهبرد‌های آنها ناشناخته است و هم از درجه اهمیت بالایی برخوردار هستند. ویژگی‌ دیگری که حوزه‌ی توجه را از سایر حوزه‌های شناختی متمایز می‌کند این است که \mls{فعالیت‌ها} و بازی‌های متعددی برای توجه و تمرکز طراحی شده و از آنها استفاده می‌شود. به همین دلیل نیازی به طراحی یک بازی جدید و صحت‌سنجی مجدد آن نیست. 
با توجه به مجموع این عوامل حوزه‌ی توجه و تمرکز انتخاب شد.\\

\section{انتخاب بازی}

بازی‌های متعددی در حوزه‌ی توجه و تمرکز توسعه پیدا کرده‌اند. این بازی‌ها روی فاکتورهای مختلف مانند توجه انتخابی، توجه تقسیم‌شده، توجه پایدار و فراخنای توجه کار می‌کنند. بازی انتخاب شده روی فاکتور توجه تقسیم شده کار می‌کند و بر تقویت توانایی انسان برای تمرکز همزمان روی چند عامل تاکید می‌نماید. هرچقدر این نوع از توجه بهتر باشد یک فرد بهتر می‌تواند چندین کار را به صورت همزمان با هم انجام دهد. 
مهم‌ترین معیار انتخاب این بود که بازی راهبرد‌های متنوع داشته باشد. به این معنا که افراد از راهبردهای مختلف برای انجام بازی استفاده کنند. با توجه به این معیار ۴ بازی به عنوان کاندید انتخاب شدند.
\subsection{بازی‌های انتخاب شده}
\textbf{بازی ابر باران‌زا:} این بازی روی توجه تقسیم‌شده کار می‌کند. بازی به این صورت است که تعدادی ابر باران‌زا و تعدادی ابر عادی در صفحه وجود دارند. این ابرها شروع به حرکت می‌کنند و کم کم ابرهای باران‌زا تبدیل به ابرهای عادی می‌شوند. در نهایت وقتی ابرها از حرکت ایستادند کاربر باید ابرهایی را که در ابتدا باران‌زا بودند مشخص کند.

\textbf{بازی سنگ-کاغذ-قیچی:} این بازی روی تصمیم‌گیری فرد کار می‌کند. این بازی مشابه بازی سنگ-کاغذ-قیچی مرسوم است با این تفاوت که کاربر با کامپیوتر بازی می‌کند و کامپیوتر برای بازی کردن الگوی مشخصی دارد. کاربر باید الگوی کامپیوتر را بفهمد و سپس با توجه به الگو به نحوی بازی کند که بیشترین امتیاز را به دست بیاورد.

\textbf{بازی مسافرخانه زنبوری:} این بازی روی حافظه کاری کار می‌کند. بازی به این صورت است که در ابتدا تعدادی کندوی زنبور عسل نمایش داده می‌شود که همه خالی هستند. سپس زنبورها در کندوها رفت و آمد می‌کنند. این رفت و آمد شامل سه حرکت است: از بیرون به داخل کندو می‌روند، از کندو خارج می‌شوند یا بین کندوها جابجا می‌شوند. پس از اینکه حرکت زنبورها به پایان رسید کاربر باید مشخص کند در هر کندو چند زنبور وجود دارد.

\textbf{کتابخانه:} این بازی روی فراخنای توجه کار می‌کند. بازی به این صورت است که تعدادی کتاب نمایش داده می‌شود. کاربر باید مشخص کند جلد هر کتاب با کتاب قبلی یکسان بوده است یا خیر. در مراحل بالاتر به جای کتاب قبلی باید دو کتاب یا سه کتاب قبلی را در نظر بگیرد.

\subsection{طراحی آزمایش برای انتخاب بازی}
برای انتخاب بازی یک آزمایش ساده طراحی شد. آزمایش به این صورت بود که آزمون‌دهنده ابتدا دستورالعمل ۴ بازی انتخاب شده را مطالعه می‌کرد و سپس یکی از آنها را انتخاب می‌کرد. هر بازی به سه بخش مجزا تقسیم شده بود که بخش اول شامل مراحل ساده، بخش دوم شامل مراحل کمی سخت‌تر و بخش سوم شامل مراحل بسیار سخت بود. از آزمون‌دهنده خواسته می‌شد که بخش‌های مختلف را به ترتیب بازی کند و پس از اتمام هر بخش راهبرد‌های مورد استفاده‌ی خود را یادداشت کند. در انتهای بازی نیز پرسشنامه‌ای راجع به ویژگی‌های فردی خود را تکمیل می‌کرد. (پرسشنامه در پیوست آمده است)
در این آزمون ۵ نفر در بازی ابرباران‌زا، ۷ نفر در بازی سنگ-کاغذ-قیچی، ۸ نفر در بازی مسافرخانه زنبوری و ۴ نفر در بازی کتابخانه شرکت کردند. (عددها رو درست کنم) در بازی ابرباران‌زا ۱۰ راهبرد، در بازی سنگ-کاغذ-قیچی ۲ راهبرد، در بازی مسافرخانه زنبوری ۳ راهبرد و در بازی کتابخانه ۴ راهبرد در مجموع گزارش شد. با توجه به نتایج به دست آمده دیده می‌شود که بازی ابرباران‌زا تنوع راهبرد بالاتری دارد و برای اهداف ما در این پژوهش مناسب‌تر است. بنابراین در نهایت بازی ابرباران‌زا انتخاب شد.

\section{طراحی آزمایش اول}
نتایج آزمایشی که در این بخش طراحی شده است پیش‌نیاز آزمایش بخش بعدی است. این بخش دو هدف دارد. اول اینکه مجموعه‌ای از راهبرد‌های مورد استفاده‌ی افراد در بازی ابر باران‌زا استخراج شود و دوم اینکه یک رده‌بندی برای راهبرد‌های موجود استخراج شود. به این معنا که مشخص شود کدام راهبرد‌(ها) به صورت میانگین کارآیی بهتری داشتند.

\subsection{ساختار آزمایش}
آزمایشی که در این بخش طراحی شد عمدتا مشابه آزمایشی بود که برای انتخاب بازی انجام دادیم ولی چند تفاوت عمده داشت. اولین تفاوت مهم این بود که بازی را به ۷ بخش تقسیم کردیم. بخش اول مراحلی بودند که یک ابر بارانی داشتند، بخش دوم دو ابر بارانی داشتند و الی آخر. آزمون به این صورت است که ابتدا آزمون‌دهنده یک بخش را کامل بازی می‌کند و سپس راهبرد‌های خود در آن بخش را یادداشت می‌کند. بازی از نظر زمانی محدود است. هر فرد ۱۰ دقیقه برای انجام بازی فرصت دارد. نحوه‌ی انتقال بین مراحل به این صورت است که اگر آزمون‌دهنده تمام ابرهای باران‌زا را به درستی تشخیص دهد به مرحله‌ی بعدی می‌رود. اما اگر حتی یکی از آنها را اشتباه انتخاب کند در همان مرحله باقی می‌ماند. محدودیت تکرار هر مرحله ۲۰ بار است. یعنی اگر فردی بعد از ۲۰ بار تکرار یک مرحله نتواند آن را با موفقیت پشت سر بگذارد امتیازی از آن مرحله به او تعلق نخواهد گرفت. بازی در مجموع ۴۲ مرحله است. تعداد مراحل در هر بخش در جدول \ref{numOfLevelTable} نمایش داده شده است.

\begin{table}[]
\centering
\caption{تعداد مراحل هر بخش}
\label{numOfLevelTable}
\begin{tabular}{|c|c|}
\hline
\textbf{تعداد مراحل} & \textbf{شماره بخش} \\ \hline
۴                    & ۱                  \\ \hline
۵                    & ۲                  \\ \hline
۶                    & ۳                  \\ \hline
۶                    & ۴                  \\ \hline
۷                    & ۵                  \\ \hline
۷                    & ۶                  \\ \hline
۷                    & ۷                  \\ \hline
\end{tabular}
\end{table}


بین هر دو بخش توقفی وجود دارد تا آزمون‌دهنده فرصت داشته باشد راهبرد‌های خود را بنویسد. در نهایت پس از اتمام زمان از آزمون‌دهنده تقاضا می‌شود پرسشنامه‌ی اطلاعات فردی را تکمیل کند.
آزمون با استفاده از نرم‌افزار  \lr{adobe flash cs6}و با استفاده از زبان برنامه‌نویسی  \lr{action script 3} طراحی شد. برای اجرای آزمون از یک لپ تاپ (\lr{Lenovo ThinkPad E460}) استفاده شد و شرکت کننده‌ها با استفاده از نشانگر جواب خود را انتخاب می‌کردند.

\subsection{ثبت داده}
در این مرحله اطلاعات را با استفاده از دو ابزار مختلف ثبت می‌کنیم. ابزار اول استفاده از اطلاعات ثبت شده از نحوه‌ی بازی کردن آزمون‌دهنده است. به ازای هر مرحله این اطلاعات شامل مکان ابرهای بارانی و عادی، تعداد ابرهایی که به درستی انتخاب شدند، تعداد ابرهایی که به اشتباه انتخاب شدند، مکان نشانگر در هر لحظه و ویژگی‌های آن مرحله از بازی است.
ابزار دیگری که برای ثبت اطلاعات از آن استفاده کردیم یک دستگاه ردیاب چشم بود. (توضیح ویژگی‌های دستگاه) هدف استفاده از این دستگاه ثبت نقطه‌ی نگاه آزمون دهنده و تطبیق آن با راهبرد‌های گزارش شده توسط وی بود.

\subsection{شرکت‌کننده‌ها}
در این مرحله، آزمون به صورت یک مسابقه برگزار شد. مجموعا ۵۷ نفر در آزمون شرکت کردند که از بین آنها اطلاعات ۴۶ نفر با توجه به پرسشنامه‌ها قابل استفاده بود. از میان این ۴۶ نفر ۱۴ نفر زن و ۳۲ نفر مرد بودند. میانگین سنی آنها ۲۳ سال با انحراف معیار ۴ بود. به عنوان جایزه به دو نفری که بیشترین امتیاز را کسب کردند یک حافظه جانبی با ظرفیت ۳۲ گیگابایت داده شد و به دو نفر نیز به قید قرعه یک حافظه جانبی با ظرفیت ۱۶ گیگابایت داده شد.

\subsection{نحوه محاسبه امتیاز}
دو معیار برای محاسبه‌ی امتیاز اهمیت دارند. اولین معیار آخرین مرحله‌ای است که شرکت‌کننده موفق شده به آن برسد و معیار دوم میزان توقف وی در مراحل دیگر است. به عنوان مثال فردی که توانسته همه‌ی مراحل را با یک بار بازی کردن پشت سر بگذارد و تا مرحله‌ی ۳۰ جلو برود باید امتیاز بیشتری از فردی بگیرد که تا مرحله‌ی ۳۰ جلو رفته اما هر مرحله را ۲ بار انجام داده است.
علاوه بر این هزینه‌ی خطاها در مراحل بالاتر بیشتر است. به این معنی که فردی که مرحله‌ی ۱ را ۵ بار تکرار کرد امتیاز بیشتری می‌گیرد نسبت به فردی که مرحله‌ی ۲۰ را ۵ بار تکرار کرده است (با فرض اینکه بقیه‌ی مراحل را مشابه هم بازی کرده باشند).
با توجه به این موضوع معیار امتیاز دهی را به این صورت تعیین کردیم که شماره‌ی مرحله ضریب امتیاز آن مرحله باشد و تعداد تکرارهای هر مرحله از ۲۱ کم می‌شود و در ضریب آن مرحله ضرب می‌شود. در نهایت امتیاز همه‌ی مراحل با هم جمع می‌شوند.

\begin{equation}
	score = \sum_{level=1}^{lastLevelReached} level(21-numOfLevelRepeat)
\end{equation}


\section{نتایج بخش اول}
در بخش اول دو هدف اصلی را دنبال می‌کردیم. هدف اول جمع‌آوری مجموعه‌ای از راهبرد‌های مورد استفاده توسط افراد بود و هدف دوم طبقه‌بندی این راهبرد‌ها بر اساس میزان موثر بودن آنها بوده است.
\subsection{راهبرد‌های استخراج شده}
در این بخش راهبرد‌های جمع‌آوری شده را  به دو دسته‌ی اصلی و فرعی تقسیم  کردیم. منظور از راهبردهای اصلی راهبرد‌هایی است که عمده‌ی عملکرد فرد تحت تاثیر آن قرار می‌گیرد و به تنهایی می‌تواند راهبرد فرد در طی آزمایش باشد. راهبردهای فرعی راهبردهایی هستند که فرد از آنها در کنار یک راهبرد اصلی استفاده می‌کند. در جدول \ref{StrategyList} لیست راهبرد‌های اصلی نمایش داده شده است. به منظور استخراج راهبرد‌ها از پرسشنامه‌هایی که توسط شرکت‌کننده‌ها تکمیل شده بود، ابتدا تمامی پرسشنامه‌ها خوانده شدند و راهبردهایی که مشابه هم بودند استخراج شدند. سپس مجددا تمامی پرسشنامه‌ها بررسی شدند و اطمینان حاصل شد که تمامی راهبرد‌هایی که نوشته شده به حداقل یک راهبرد استخراج شده مرتبط می‌شود. 
\begin{table}[]
	\centering
	\caption{راهبرد‌های اصلی}
	\label{StrategyList}
	\begin{scriptsize}
	\begin{center}
	\renewcommand{\arraystretch}{2}
	\begin{tabular}{|c|c|r|}
		\hline
\textbf{شماره گروه} & \textbf{شماره راهبرد} & \multicolumn{1}{c|}{\textbf{توضیح راهبرد}} \\ \hline
		\multirow{4}{*}{۱} & 1 & دنبال کردن ابر با چشم \\ \cline{2-3} 
		& 2 & دنبال کردن ابر با استفاده از ماوس \\ \cline{2-3} 
		& 3 & دنبال کردن ابر با استفاده از انگشتان دست \\ \cline{2-3} 
		& 4 & سوئیچ کردن نگاه بین ابرها \\ \hline
		\multirow{5}{*}{۲} & 5 & نگاه کردن به مرکز صفحه یا مرکز ابرهای باران‌زا یا نگاه کلی به صفحه (نگاه کردن کل ابرها به صورت همزمان) \\ \cline{2-3} 
		& 6 & نگاه کردن به یک ابر باران‌زا در حالی که سایر ابرها در دامنه دید هستند \\ \cline{2-3} 
		& 7 & سوئیچ کردن نگاه بین مرکز دو دسته ابر باران‌زا \\ \cline{2-3} 
		& 8 & تصور کردن به صورت خط یا شکل هندسی \\ \cline{2-3} 
		& 9 & دنبال کردن برخی از ابرها با یک چشم و برخی دیگر با چشم دیگر \\ \hline
		\multirow{4}{*}{۳} & 10 & توجه بیشتر به ابرهای نواحی شلوغ \\ \cline{2-3} 
		& 11 & توجه بیشتر به ابرهایی که سرعت و دامنه حرکت بیشتری دارند \\ \cline{2-3} 
		& 12 & توجه بیشتر به نواحی که ابرهای باران‌زای بیشتری دارند \\ \cline{2-3} 
		& 13 & توجه بیشتر به ابرهایی که در یک جهت حرکت می‌کردند \\ \hline
	\end{tabular}
	\end{center}
	\end{scriptsize}
\end{table}

راهبرد‌های جدول \ref{StrategyList} که در یک دسته قرار گرفته‌اند ویژگی‌های مشابه دارند. این ویژگی‌ها در جدول \ref{mainStrategyGroups} نمایش داده شده‌اند. در نهایت دسته‌ها با یکدیگر مقایسه شده‌اند.

\begin{table}[]
\centering
\caption{ویژگی‌های مشترک هر دسته از راهبرد‌ها}
\label{mainStrategyGroups}
\begin{scriptsize}
\begin{center}
\renewcommand{\arraystretch}{2}
\begin{tabular}{|c|r|}
\hline
\textbf{شماره گروه} & \multicolumn{1}{c|}{\textbf{ویژگی مشترک}} \\ \hline
۱ & نقطه تمرکز چشم در هر لحظه روی یک ابر بارانی است \\ \hline
۲ & نقطه تمرکز چشم در هر لحظه روی هیچ کدام از ابرهای بارانی نیست \\ \hline
۳ & نقطه تمرکز چشم بعضی اوقات روی یکی از ابرها و بعضی اوقات در نقطه‌ای خارج از ابرهای بارانی است. \\ \hline
\end{tabular}
\end{center}
\end{scriptsize}
\end{table}
در جدول \ref{secondaryStrategies} لیست راهبرد‌های فرعی نمایش داده شده‌اند.

\begin{table}[]
\centering
\caption{راهبرد‌های فرعی}
\label{secondaryStrategies}
\begin{scriptsize}
\begin{center}
\renewcommand{\arraystretch}{2}
\begin{tabular}{|c|r|}
\hline
\textbf{شماره راهبرد} & \multicolumn{1}{c|}{\textbf{توضیح راهبرد}} \\ \hline
۱ & جدا کردن یک یا چند ابر و دنبال کردن آن با گوشه چشم (دامنه بینایی) یا ماوس \\ \hline
۲ & صرف نظر کردن از تعدادی از ابرها \\ \hline
۳ & پیش‌بینی حرکت برخی از ابرها \\ \hline
۴ & افزایش توجه هنگام کند شدن حرکت ابرها \\ \hline
۵ & ثبت یک تصویر ذهنی از مکان ابرها هنگامی که رنگشان تغییر می‌کند \\ \hline
۶ & تنگ‌تر کردن چشم \\ \hline
\end{tabular}
\end{center}
\end{scriptsize}
\end{table}


\subsection{امتیازدهی به راهبرد‌ها}
به منظور امتیازدهی به راهبرد‌ها ابتدا امتیاز هر  \mls{شرکت‌کننده} در هر بخش را محاسبه کردیم. روش محاسبه‌ی امتیاز در هر بخش مشابه روش محاسبه‌ی امتیاز کل هر شرکت‌کننده بود با این تفاوت که به جای اینکه تمامی مراحل در امتیاز دهی دخیل باشند تنها مراحل همان بخش در امتیازدهی دخیل بودند. با توجه به اینکه شماره مراحل بخش‌های آخر بیشتر از شماره مراحل بخش‌های اول بودند امتیاز مراحل آخر نیز از سطح بالاتری شروع می‌شدند. به عنوان مثال کسی که یک مرحله از بخش ۷ را انجام دهد امتیاز بیشتری از بخش ۷ می‌گیرد نسبت به کسی که یک مرحله از بخش ۲ را انجام می‌دهد. امتیاز هر مرحله با استفاده از رابطه‌ی \ref{levelScoreEq} محاسبه می‌شود.
\begin{equation}
\label{levelScoreEq}
	score(person = i, part = j) = \sum_{level = FirstLevel(j)}^{LastLevel(j)} level(21-numOfLevelRepeat(level, i))
\end{equation}

هدف نهایی این بخش این است که بفهمیم هر راهبرد در هر بخش به صورت میانگین چقدر امتیاز برای شرکت‌کننده‌گان کسب کرده است. راهبرد‌ی که موفق شده باشد میانگین امتیاز بالاتری کسب کند راهبرد برنده در آن بخش است. به همین منظور پس از اینکه امتیاز هر بخش محاسبه شد با استفاده از آن امتیاز هر راهبرد را محاسبه می‌کنیم. برای این کار از رابطه‌ی \ref{StrategyScoreEq} استفاده می‌کنیم. در این رابطه n  تعداد افراد شرکت‌کننده در آزمایش است. 

\begin{equation}
\label{UseStrategyEq}
useStrategy(person = i, strategy = k, part = j) =\begin{cases}
    1 ,& \text{\lr{if person i used strategy k in part j}} \\
    0 , & \text{otherwise}
    \end{cases}
\end{equation}


\begin{equation}
count(strategy = k, part = j) = \sum_{i=1}^{n} useStrategy(i, k, j)
\end{equation}

\begin{equation}
\label{StrategyScoreEq}
	score(strategy = k, part = j) = \frac{1}{count(k, j)} \sum_{i=1}^{n} score(person = i, part = j) useStrategy(i, k, j)
\end{equation}


امتیاز هر راهبرد بر اساس این روابط میانگین امتیازهایی است که هر فرد با استفاده از این راهبرد به دست آورده است. در شکل   \ref{fig:part2to5avgscores} می‌توانیم این میانگین‌ها را در بخش‌های مختلف ببینیم.

\begin{figure}
\centering
\includegraphics[scale=0.8]{Figures/part2-5AvgScores.png}
\caption{\label{fig:part2to5avgscores}
میانگین امتیاز هر گروه راهبرد در بخش‌های مختلف
}
\end{figure}




همون طور که در شکل \ref{fig:part2to5avgscores} دیده می‌شود میانگین راهبردهای گروه ۲ نسبت به گروه ۱ و ۳ بیشتر است. در ادامه قصد داریم امتیاز راهبردهای مختلف را به صورت آماری با هم مقایسه کنیم. برای این کار از آزمون آماری t  استفاده می‌کنیم که در آن مقدار حداقل سطح معناداری برابر با  $\alpha = 0.5$ در نظر گرفته شده است. در جدول \ref{tbl:StttestRes} مقادیر p-value به ازای هر دو راهبرد در مقایسه با یکدیگر نمایش داده شده است. همانطور که معلوم است در بخش ۵ و ۶ راهبردهای گروه ۲ برتری معناداری نسبت به راهبردهای گروه ۱ و ۳ دارند.

\begin{table}[]
\begin{scriptsize}
\begin{center}
\centering
\caption{نتایج به دست آمده از آزمون آماری t ، مقدار هر خانه p-value مربوط به مقایسه‌ی دو استراتژی نمایش داده شده در ستون سمت راست در هر بخش را نمایش می‌دهد. خانه‌هایی که رنگشان نارنجی است خانه‌هایی هستند که مقدار p-value در آنها معنادار است.}
\label{tbl:StttestRes}
\renewcommand{\arraystretch}{2}
\begin{tabular}{|c|c|c|c|c|c|}
\hline
\multicolumn{1}{|l|}{}                                                          & \textbf{بخش ۲}               & \textbf{بخش ۳} & \textbf{بخش ۴} & \textbf{بخش ۵}               & \textbf{بخش ۶}               \\ \hline
\textbf{\begin{tabular}[c]{@{}c@{}}گروه ۱ در\\   مقایسه با گروه ۲\end{tabular}} & ۰/۱۲                         & ۰/۲۱           & ۰/۱۷           & \cellcolor[HTML]{FFCE93}۰/۰۴ & \cellcolor[HTML]{FFCE93}۰/۰۱ \\ \hline
\textbf{\begin{tabular}[c]{@{}c@{}}گروه ۱ در\\   مقایسه با گروه ۳\end{tabular}} & ۰/۰۸                         & ۰/۴۰           & ۰/۲۷           & ۰/۲۰                         & ۰/۴۳                         \\ \hline
\textbf{\begin{tabular}[c]{@{}c@{}}گروه ۲ در\\   مقایسه با گروه ۳\end{tabular}} & \cellcolor[HTML]{FFCE93}۰/۰۱ & ۰/۳۰           & ۰/۳۲           & ۰/۱۱                         & \cellcolor[HTML]{FFCE93}۰/۰۱ \\ \hline
\end{tabular}
\end{center}
\end{scriptsize}
\end{table}
برای تحلیل راهبردهای فرعی نیز از روش مشابهی استفاده کردیم ولی به علت محدود بودن داده‌های مربوط به راهبردهای فرعی هیچ تفاوت معناداری بین آنها پیدا نشد. جدول ... نمره‌ی این راهبردها را نمایش می دهد. در هر بخش برخی از این راهبردها نمره‌ی بیشتری کسب کرده‌اند ولی هیچ کدام از آنها نسبت به دیگری از لحاظ آماری برتری نداشتند. میانگین امتیاز این راهبردها در شکل \ref{fig:notMainStrAvg} نمایش داده شده است.

\begin{figure}
\centering
\includegraphics[scale=0.8]{Figures/notMainStrAvg.png}
\caption{\label{fig:notMainStrAvg}
میانگین امتیاز راهبردهای فرعی در بخش‌های مختلف
}
\end{figure}


\subsection{راهبردهای منتخب برای آزمایش دوم}

نتایج به دست آمده از بخش قبل نشان می‌دهد که گروه دوم راهبردها به نسبت گروه ۱ و ۳ میانگین امتیاز بیشتری در برخی از بخش‌ها دارند. با توجه به اینکه اکثر افراد تا بخش ۴ به راحتی جلو می‌آیند و از این نقطه به بعد است که با مشکل مواجه می‌شوند راهبرد موثر در بخش ۴ و ۵ و ۶ اهمیت بیشتری پیدا می‌کند. با توجه به اینکه میانگین نمره‌ی گروه راهبرد ۲ تقریبا در تمام بخش‌ها بیشتر از سایر گروه‌ها است و در بخش ۵ و ۶ تفاوت معناداری با سایر گروه‌ها دارد این گروه راهبرد برای آموزش در بخش بعدی انتخاب می‌شود.

\subsection{همبستگی با ویژگی های فردی}

\subsection{مشکلات موجود در آزمایش اول}

مشکلاتی در آزمایش اول وجود داشتند که در این بخش به آنها می‌پردازیم. اولین موضوع کم بودن تعداد افرادی بود که در آزمایش شرکت کردند. به خصوص با توجه به طیف گسترده‌ی راهبردها و روش‌های مختلفی که افراد حین بازی کردن از آن استفاده می‌کردند کم بودن افراد باعث شد در بخشی از قسمت‌ها اطلاعات کافی برای تحلیل نداشته باشیم. به عنوان مثال در مورد استراتژی‌های فرعی کمبود داده باعث شد نتوانیم تحلیل جامعی ارائه دهیم. پیشنهادی که برای آینده می‌توان ارائه داد این است که این تست به صورت آنلاین برگزار شود و افراد راهبردهای مورد استفاده‌ی خود را به صورت گزینه‌ای گزارش کنند تا امکان جمع‌آوری اطلاعات در اندازه‌ی بالا ایجاد شود.

مشکل دیگر کوتاه بودن مدت زمان آزمایش بود. افراد ۱۰ دقیقه فرصت داشتند که بازی کنند و راهبردهای خود را بنویسند. این موضوع باعث می‌شد فرصت اندکی برای حرفه‌ای شدن در بازی و رسیدن به راهبردهای موثر وجود داشته باشد. پیشنهاد می‌شود در پژوهش‌های آتی از یک دوره‌ی تمرینی استفاده شود. به این صورت که افراد حدودا یک ماه فرصت داشته باشند که بازی کنند و در طی این مدت راهبردهای خود را به مرور به روز کنند.

\chapter{آزمایش دوم - انتقال راهبرد}
\label{chapter:PartTwoTransfer}
\thispagestyle{plain}

\section{مقدمه}
در این بخش هدف اصلی بررسی تاثیر انتقال راهبرد بر عملکرد بازیکن است. به این منظور از نتایج به دست آمده در بخش قبل استفاده می‌کنیم و تلاش می‌کنیم راهبرد انتخاب شده را با استفاده از دو روش به بازیکن منتقل کنیم. علاوه بر این موضوع دیگری که در این بخش مورد بررسی قرار می‌گیرد رابطه‌ی میان مدل تصمیم‌گیری بازیکن (مبتنی بر مدل بودن یا غیر مبتنی بر مدل بودن) و میزان یادگیری راهبرد است.

\section{طراحی آزمایش دوم}
\subsection{ساختار آزمایش اصلی} \label{partTwoMainTest}

این آزمایش از ۳ بخش تشکیل شده است.
\subsubsection{بخش اول}
در بخش اول افراد در \mls{آزمون دو مرحله‌ای دا} که در \cite{daw2011ModelBased} توضیح داده شده است، شرکت کردند. این آزمون از دو بخش تمرینی و اصلی تشکیل شده است. بخش تمرینی شامل ۴۰ دور و بخش اصلی شامل ۲۰۰ دور است. در هر دور افراد ابتدا تصویر دو هواپیما را مشاهده می‌کنند سپس ۲ ثانیه فرصت دارند که با استفاده از کلید f یا j یکی از این دو تصویر را انتخاب کنند. پس از انتخاب کردن یکی از این دو تصویر به یکی از دو جنگل موجود راهنمایی می‌شوند. دو تصویر از جنگل انتخاب شده نمایش داده می‌شود. مجددا بازیکن باید با استفاده از کلید f  یا j یکی از دو تصویر را انتخاب کند. پس از انتخاب تصویر به او نمایش داده می‌شود که در آن جنگل گنج وجود دارد یا خیر. اگر گنج وجود داشته باشد امتیاز می‌گیرد و اگر وجود نداشته باشد امتیازی نمی‌گیرد.
احتمال رفتن به هر جنگل پس از انتخاب هر هواپیما در شکل \ref{fig:plainJungleProb} مشخص شده است. با انتخاب هواپیمای سفید با احتمال ۳۰ درصد به جنگل سمت راست و با احتمال ۷۰ درصد به جنگل سمت چپ می‌رود و با انتخاب هواپیمای نارنجی احتمال‌های دقیقا برعکس می‌شوند.
احتمال وجود داشتن گنج در هر جنگل از توزیع \mls{قدم زدن تصادفی} پیروی می‌کند. بنابراین احتمال وجود داشتن گنج در هر جنگل در حال تغییر است و بازیکن در حین بازی با امتحان کردن حالت‌های مختلف باید تلاش کند بیشترین امتیاز را کسب کند.
\begin{figure}
\centering
\includegraphics[scale=0.8]{Figures/plainJungleProb.png}
\caption{\label{fig:plainJungleProb}
احتمال منتقل شدن از هر هواپیما به هر جنگل
}
\end{figure}

\subsubsection{بخش دوم}

در بخش دوم، از بازی «ابر باران‌زا» که در آزمایش اول از آن استفاده شده بود، استفاده می‌شود منتها ساختار آزمایش متفاوت است. در این بخش آزمایش به این صورت طراحی شده است که ابتدا ۴ مرحله‌ی آزمایشی اجرا می‌شود که ۱ یا ۲ ابر باران‌زا دارند. این چهار مرحله به منظور آشنایی بازیکن با ساختار و نحوه‌ی انجام بازی قرار داده شده‌اند. سپس بازی اصلی شروع می‌شود. بازی اصلی شامل ۵ بخش است. بخش اول شامل ۳ ابر باران‌زا است (به همین دلیل بخش ۳ نامیده شده است) و به ترتیب بخش‌های بعدی شامل ۴و ۵ و ۶و ۷ ابر باران‌زا هستند. بر خلاف آزمایش اول که محدودیت زمانی داشت در این آزمایش محدودیت زمانی نداریم. ساختار بازی به این صورت است که هر بخش شامل ۱۰ مرحله است. در هر مرحله اگر بازیکن موفق شود تمامی ابرها را به درستی تشخیص دهد امتیاز آن مرحله را کامل دریافت می‌کند اما اگر حتی یکی از ابرها را درست تشخیص ندهد هیچ امتیازی از آن مرحله کسب نخواهد کرد. در انتهای هر بخش درصد مراحلی که با موفقیت گذرانده محاسبه می‌شود. برای هر بخش یک حد نصاب در نظر گرفته شده است که اگر بازیکن بتواند حد نصاب را کسب کند می‌تواند وارد بخش بعدی شود. اگر موفق نشود می‌تواند تا زمانی که بخواهد آن بخش را مجددا بازی کند تا جایی که به این نتیجه برسد که نمی‌تواند این درصد را بهبود ببخشد و انتخاب می‌کند که بازی تمام شود. حد نصاب بخش‌های مختلف در جدول \ref{tbl:quorum} نمایش داده شده است. 

\begin{table}[]
\begin{scriptsize}
\begin{center}
\centering
\caption{حد نصاب بخش‌های مختلف برای رفتن به بخش بعدی در آزمایش دوم}
\label{tbl:quorum}
\renewcommand{\arraystretch}{2}
\begin{tabular}{|c|c|}
\hline
\textbf{شماره بخش}& \textbf{حد نصاب}\\ \hline
\textbf{\begin{tabular}[c]{@{}c@{}}بخش ۳ به ۴\end{tabular}} & موفقیت در حداقل ۸۰ درصد مراحل\\ \hline
\textbf{\begin{tabular}[c]{@{}c@{}}بخش ۴ به ۵\end{tabular}} & موفقیت در حداقل ۶۰ درصد مراحل\\ \hline
\textbf{\begin{tabular}[c]{@{}c@{}}بخش ۵ به ۶\end{tabular}} & موفقیت در حداقل ۵۰ درصد مراحل\\ \hline
\textbf{\begin{tabular}[c]{@{}c@{}}بخش ۶ به ۷\end{tabular}} & موفقیت در حداقل ۵۰ درصد مراحل\\ \hline
\end{tabular}
\end{center}
\end{scriptsize}
\end{table}

حدنصاب‌های به دست آمده با استفاده از اطلاعات جمع‌آوری شده در آزمایش اول استخراج شده‌اند. به این صورت که تعداد مراحلی که هر بازیکن توانسته در هر بخش به صورت موفقیت آمیز انجام بدهد محاسبه شده و سپس بین تمامی افراد میانگین گرفته شده است. پس از اتمام هر بخش از بازیکن خواسته می‌شود راهبرد مورد استفاده‌ی خود را یادداشت کند.  

\subsubsection{بخش سوم} \label{partTwoMainTest:three}

پس از اتمام دور اول بازی، راهبرد انتخاب شده در آزمایش اول به صورت کلامی به بازیکن آموزش داده می‌شود. در این راهبرد از بازیکن خواسته می‌شود در حین انجام آزمایش به مرکز ابرهای باران‌زا نگاه کند در حالی که ابرها در دامنه‌ی دیدش هستند. یا به بیان دیگر یک چند ضلعی فرضی در نظر بگیرد و مرکز آن را نگاه کند. علاوه بر آموزش کلامی یک راهنما نیز در بازی قرار داده شده است تا بازیکن با استفاده از آن راهبرد جدید را بیاموزد. این راهنما به این صورت عمل می‌کند که در ۵ مرحله‌ی ابتدایی هر بخش بازیکن به صورت اختیاری امکان استفاده از راهنما را خواهد داشت. اگر گزینه‌ی استفاده از راهنما را انتخاب کند نقطه‌ی مرکزی ابرهای باران‌زا و \mls{بدنه‌ی محدب} آنها نمایش داده می‌شود. در حین کند شدن حرکت ابرها راهنما نیز کمرنگ می‌شود تا نهایتا حذف شود. از بازیکن‌ها خواسته می‌شود پس از اتمام هر بخش میزان استفاده‌ی خود از راهبرد آموزش داده شده را توسط عددی بین ۱ تا ۵ مشخص کنند. ۱ به این معنا است که در هیچ کدام از مراحل نتوانسته‌اند از راهبرد استفاده کنند و ۵ به این معنا است که در تمامی مراحل موفق به استفاده از راهبرد شده‌اند. در این بخش نیز مانند بخش دوم بازی بازیکن می‌تواند هر بخش را مطابق تشخیص خود تکرار کند.

\subsubsection{پس از اتمام}

پس از اتمام آزمایش به منظور درک دقیق‌تر راهبردهای استفاده شده شرکت‌کننده، لیستی از راهبردهای جمع‌آوری شده به بازیکن ارائه می‌شود و از او خواسته می‌شود راهبردهایی که خود نوشته است با راهبردهای این لیست تطبیق بدهد. در نهایت با پر کردن فرم اطلاعات فردی آزمایش به پایان می‌رسد. پانزده هزار تومان بابت شرکت کردن در آزمایش و پانزده هزار تومان به نسبت عملکرد هر فرد به او داده می‌شود.
طراحی ساختار آزمایش به شکل کنونی سه دلیل اصلی دارد. اول اینکه محدودیت زمانی باعث کاهش کارآیی فرد نشود. دوم اینکه می‌خواهیم بازیکن به بهترین عملکرد خود برسد. در واقع کمبود تمرین باعث محدود شدن کارآیی‌اش نشود. دلیل سوم که در ادامه‌ی دلیل دوم است این است که میخواهیم فرد به مرحله‌ای برسد که با تمرین بیشتر عملکردش بهبود پیدا نکند. در واقع به نقطه‌ای برسد که حد توانمندی فردی اوست.

\subsection{گروه کنترلی اول}

در راستای مقایسه‌ی نتایج به دست آمده از آزمایش اصلی، یک آزمایش دیگر انجام شد که در آن راهبرد موثر به افراد آموزش داده نمی‌شود. ساختار آزمایش دو تفاوت با آزمایش اصلی دارد. اول اینکه آزمون مدل محور یا مدل آزاد بودن یادگیری فرد حذف شد و دوم اینکه آموزش راهبرد انجام نمی‌شود. سایر بخش‌های آزمایش مشابه آزمایش اصلی است.

\subsection{گروه کنترلی دوم}

موضوعی که باعث نیاز به گروه کنترلی دوم می‌شود این است که ممکن است گزارش دادن راهبرد روی عملکرد افراد تاثیر بگذارد. بنابراین نیاز به آزمایش دیگری داریم که افراد در آن درگیر گزارش راهبرد نشوند. این آزمایش به این صورت است که افراد دو بار بازی «ابرهای باران‌زا» را انجام می‌دهند بدون اینکه هیچ راهبردی گزارش کنند و در نهایت فرم اطلاعات فردی را تکمیل می‌کنند.

\subsection{شرکت‌کننده‌ها}

در آزمایش اصلی ۲۲ نفر شرکت کردند که ۷ نفر آنها زن و ۱۵ نفر مرد هستند. میانگین سنی آنها ۲۴ سال با انحراف معیار ۳ است. در آزمایش کنترلی اول ۲۰ نفر شرکت کردند که ۵ نفر آنها زن و ۱۵ نفر مرد هستند. میانگین سنی آنها ۲۶ سال با انحراف معیار ۲ است. 

\section{نتایج بخش دوم}

\subsection{مولفه‌های استخراج شده از داده‌ها}
در این بخش مولفه‌های استخراج شده از داده‌های به دست آمده از بازی‌ها را بررسی می‌کنیم و هر کدام را جداگانه تعریف می‌کنیم.
\subsubsection{مولفه‌های مربوط به مدل محور یا مدل آزاد بودن}

آزمون دا چندین مولفه را معرفی می‌کند. ما در این پژوهش از دو مولفه‌ی اصلی استفاده می‌کنیم: w و $\beta$. مولفه w عددی بین صفر و یک است و نشان‌دهنده‌ی میزان مدل محور بودن فرد است. هرچقدر w نزدیک‌تر به صفر باشد به این معنا است که فرد بیشتر مدل آزاد است و هر چقدر نزدیک به یک باشد به معنای مدل محور بودن فرد است. $\beta$ مولفه‌ی \mls{دمای معکوس} است. هرچه این مولفه مقدار کمتری داشته باشد به این معناست که فرد تمایل بیشتری به امتحان کردن انتخاب‌های جدید دارد و هرچه مقدار آن بیشتر باشد به این معناست که فرد تمایل بیشتری به ادامه‌ی انتخاب‌های پیشین خود دارد. در ادامه ارتباط این دو مولفه را با میزان استفاده‌ی بازیکن‌ها از راهبردها بررسی خواهیم کرد. 


\subsubsection{مولفه‌ی مربوط به میزان پیشرفت بازیکن}
همانطور که در بخش \ref{partTwoMainTest} مشخص شد بازی ۵ بخش اصلی دارد که هر کدام شامل ۱۰ مرحله هستند. بازیکن‌ها دو دور بازی می‌کنند. برای محاسبه‌ی درصد موفقیت فرد در هر بار بازی کردن از رابطه‌ی \ref{eq:partScoreEq} استفاده می‌کنیم. در این رابطه درصد وزن‌دار امتیاز کسب شده از بازی محاسبه می‌شود. به این صورت که بیشترین امتیازی که یک فرد می‌تواند از هر ۵ بخش کسب کند ۲۵۰ امتیاز است (در صورتی که تمامی مراحل تمامی بخش‌ها را با موفقیت انجام دهد) و امتیازی که کسب کرده حاصل جمع وزن‌دار امتیازش در هر بخش است. ضریب امتیاز هر بخش شماره‌ی همان بخش است. به این معنا که ۱ امتیاز بیشتر در بخش هفتم ارزش بیشتری از ۱ امتیاز بیشتر بخش سوم دارد. حاصل تقسیم این دو عدد درصد موفقیت فرد را مشخص می‌کند.

\begin{equation}
\label{eq:partScoreEq}
	gameScore = \frac{\sum_{i=3}^{7} numOfWonLevels(i)*i }{250}
\end{equation}

برای محاسبه‌ی پیشرفت بازیکن ابتدا درصد موفقیتش در هر دور از بازی را محاسبه می‌کنیم. سپس درصد پیشرفت وی را با استفاده از رابطه‌ی \ref{eq:improveEq} محاسبه می‌کنیم. این درصد در آزمایش اصلی و کنترلی به همین شکل محاسبه شده است.

\begin{equation}
\label{eq:improveEq}
	scoreChangePercent(score1, score2) = \frac{score2 - score1}{score1}*100
\end{equation}

\subsubsection{مولفه مربوط به میزان استفاده از راهبرد}

همان‌طور که در بخش \ref{partTwoMainTest:three} گفته شد بازیکن‌ها میزان استفاده‌ی خود از راهبرد را در هر بخش توسط عددی بین ۱ تا ۵ گزارش می‌کنند. برای محاسبه‌ی میزان استفاده‌ی هر بازیکن از راهبردها می‌خواهیم درصد این استفاده را محاسبه کنیم. برای این کار از رابطه‌ی \ref{eq:strategyUsePercent} استفاده می‌کنیم. در این رابطه lastPart آخرین بخشی است که بازیکن موفق شده به آن برسد.
\begin{equation}
\label{eq:strategyUsePercent}
strategyUsePercent = \frac{\sum_{i=3}^{lastPart} useScore(i)}{lastPart*5}
\end{equation}

بنابراین هرچقدر بازیکن بیشتر موفق شده باشد در بخش‌هایی که بازی کرده از راهبرد استفاده کند این عدد نزدیک‌تر به ۱ خواهد بود.

\subsubsection{مولفه‌ی مربوط به زمان پاسخگویی}
علاوه بر میزان پیشرف بازیکن زمان پاسخگویی وی نیز می‌تواند حامل اطلاعات ارزشمندی باشد. به این منظور قصد داریم تغییرات مربوط به زمان پاسخگویی وی از دور اول بازی به دور دوم را بررسی کنیم. برای رسیدن به این هدف نیاز داریم برای هر دور از بازی میانگین زمان پاسخگویی را محاسبه کنیم. زمان پاسخگویی هر مرحله اختلاف بین زمان انتخاب آخرین ابر و اولین ابر است. بنابراین زمان پاسخگویی هر مرحله معلوم است. برای اینکه یه مقدار کلی برای یک دور بازی به دست بیاوریم از زمان پاسخگویی تمام مراحلی که بازیکن بازی کرده است میانگین می‌گیریم. بنابراین برای هر دور از بازی یک عدد به عنوان میانگین زمان پاسخگویی خواهیم داشت. با استفاده از رابطه‌ی \ref{eq:responseTimeDiffPercent} درصد تغییرات این میانگین را محاسبه می‌کنیم.

\begin{equation}
\label{eq:responseTimeDiffPercent}
	responseTimeChangePercent(t1, t2) = \frac{t2 - t1}{t1}*100
\end{equation}

\subsection{تحلیل همبستگی}
در این بخش قصد داریم میزان همبستگی بین مولفه‌های مختلف را بررسی کنیم. برای این کار از ضریب همبستگی  \mls{پیرسون} استفاده می‌کنیم.
\subsubsection{تحلیل همبستگی بین پیشرفت بازیکن و استفاده از راهبرد}

به منظور بررسی میزان همبستگی بین دو مولفه ابتدا \mls{نمودار نقطه‌ای} آنها را رسم می‌کنیم. این نمودار در شکل \ref{fig:impv1strUseScatter} نمایش داده شده است.

\begin{figure}
\centering
\includegraphics[scale=0.8]{Figures/impv1strUseScatter.png}
\caption{\label{fig:impv1strUseScatter}
نمودار نقطه‌ای پیشرفت بازیکن در برابر میزان استفاده‌ی وی از راهبرد
}
\end{figure}

سپس با استفاده از روش پیرسون ضریب همبستگی را حساب می‌کنیم. نتیجه‌ی به دست آمده نشان میدهد ضریب همبستگی برابر با ۰/۰۹ است و عدد p-value برابر با ۰/۶ است. این نتایج نشان می‌دهد این دو مولفه همبستگی معناداری با هم ندارند.

\subsubsection{تحلیل همبستگی بین استفاده از راهبرد و مولفه w }

مولفه‌ی w نشان می‌دهد بازیکن چه مقدار مدل محور است. هرچقدر این مولفه نزدیکتر به یک باشد به این معنا است که بازیکن بیشتر مدل محور است. مشابه بخش قبل از روش پیرسون برای محاسبه‌ی ضریب همبستگی استفاده می‌کنیم. نمودار نقطه‌ای این دو مولفه در شکل \ref{fig:stUsewScatter} نمایش داده شده است.

\begin{figure}
\centering
\includegraphics[scale=0.8]{Figures/stUsewScatter.png}
\caption{\label{fig:stUsewScatter}
نمودار نقطه‌ای مولفه‌ی w هر بازیکن در برابر میزان استفاده‌ی وی از راهبرد
}
\end{figure}

در این حالت ضریب همبستگی پیرسون برابر با ۰/۰۱- و عدد p-value برابر با ۰/۹ است که نشان می‌دهد این دو مولفه همبستگی معناداری با هم ندارند.

با توجه به نمودار \ref{fig:stUsewScatter} می‌توانیم مشاهده کنیم در تعداد زیادی از نقاط مقدار w بسیار نزدیک به صفر است. انتظار داریم افرادی که در آزمون دا نمره‌ای نزدیک به صفر می‌گیرند افرادی باشند که مدل آزاد هستند ولی حالت دیگری که ممکن است اتفاق بیافتد این است که مولفه w افرادی که از یادگیری تقویتی استفاده نکرده‌اند یا روش انجام دادن آزمون را به درستی متوجه نشده‌اند نیز نزدیک به صفر می‌شود. بنابراین تحلیل همبستگی را یک بار دیگر بعد از حذف داده‌هایی که w آنها صفر است تکرار می‌کنیم. نمودار نقطه‌ای نتیجه‌ی به دست آمده در شکل \ref{fig:stUsewScatterNoZero} نمایش داده شده است. در این حالت ضریب همبستگی برابر با ۰/۱۴- و عدد p-value برابر با ۰/۶ است که نشان می‌دهد در این حالت نیز همبستگی معناداری بین دو مولفه وجود ندارد.

\begin{figure}
\centering
\includegraphics[scale=0.8]{Figures/stUsewScatterNoZero.png}
\caption{\label{fig:stUsewScatterNoZero}
نمودار نقطه‌ای مولفه‌ی w هر بازیکن در برابر میزان استفاده‌ی وی از راهبرد بعد از حذف w های صفر
}
\end{figure}

\subsubsection{تحلیل همبستگی بین استفاده از راهبرد و مولفه $\beta$}

مولفه‌ی $\beta$ معادل دمای معکوس است. هرچه دما بیشتر باشد $\beta$ کمتر است و به این معنا است که تمایل بازیکن به انتخاب گزینه‌های جدید و کاوش کردن محیط بیشتر است. هرچه دما کمتر باشد $\beta$ بیشتر است و به این معنا است که تمایل بازیکن به حفظ انتخاب‌های قبلی خود بیشتر است. می‌خواهیم میزان همبستگی $\beta$ با میزان استفاده از راهبرد را بسنجیم. در شکل \ref{fig:betaStUseScatter} نمودار نقطه‌ای این دو مولفه نمایش داده شده است. ضریب همبستگی پیرسون در این حالت برابر با ۰/۰۴ و عدد p-value برابر با ۰/۸۴ است که نشان می‌دهد این دو مولفه همبستگی معناداری با هم ندارند.

\begin{figure}
\centering
\includegraphics[scale=0.8]{Figures/betaStUseScatter.png}
\caption{\label{fig:betaStUseScatter}
نمودار نقطه‌ای مولفه $\beta$ هر بازیکن در برابر میزان استفاده‌ی وی از راهبرد
}
\end{figure}

\subsubsection{تحلیل همبستگی بین استفاده از راهبرد و تغییرات میانگین زمان پاسخگویی}

برای تحلیل این قسمت میانگین زمان پاسخگویی را به دو بخش تقسیم کردیم. افرادی که میانگین زمان پاسخگویی‌شان کاهش پیدا کرده و در نتیجه تغییرات میانگین زمان پاسخگویی‌شان منفی است و افرادی که میانگین زمان پاسخگویی‌شان افزایش پیدا کرده و در نتیجه تغییرات میانگین زمان پاسخگویی‌شان مثبت است. از بین ۲۲ نفری که در آزمایش شرکت کردند ۱۷ نفر کاهش میانگین زمان پاسخگویی و ۵ نفر افزایش میانگین زمان پاسخگویی داشتند. با توجه به اینکه تعداد افرادی که افزایش میانگین زمان پاسخگویی داشتند خیلی کم است نتایج به دست آمده از تحلیل همبستگی آنها معنادار نیست. نمودار نقطه‌ای افرادی که میانگین زمان پاسخ‌شان منفی است در شکل \ref{fig:stUseDtScatter} نمایش داده شده است. در این حالت ضریب همبستگی معادل ۰/۴۸ و عدد p-value برابر با ۰/۰۴۹ است. این مقادیر نشان می‌دهند این دو مولفه با هم همبستگی معناداری دارند.

\begin{figure}
\centering
\includegraphics[scale=0.8]{Figures/stUseDtScatter.png}
\caption{\label{fig:stUseDtScatter}
نمودار نقطه‌ای مولفه $\beta$ هر بازیکن در برابر میزان استفاده‌ی وی از راهبرد
}
\end{figure}



\subsection{تحلیل آماری اختلاف میانگین‌ها}
در این قسمت قصد داریم مولفه‌های مختلف را بین گروه‌های مختلف آزمایش با هم مقایسه کنیم. به این منظور از آزمون آماری t  استفاده می‌کنیم که در آن مقدار حداقل سطح معناداری برابر با  $\alpha = 0.5$ در نظر گرفته شده است. به این منظور در هر بخش ابتدا باید بررسی کنیم شرایط آزمون t برقرار باشد. این شرایط عبارت هستند از جمع‌آوری داده‌ها به صورت تصادفی صورت گرفته باشد، هر مشاهده‌ای مستقل از سایر مشاهدات باشد و توزیع نمونه نرمال یا تقریبا نرمال باشد. دو شرط اول برای تمامی داده‌ها صدق می‌کند بنابراین شرط نرمال بودن توزیع را در هر بخش جداگانه بررسی خواهیم کرد. در صورتی که توزیع‌ها از نرمال خیلی فاصله داشته باشد از آزمون \mls{ویلکاکسون} استفاده خواهیم کرد.

\subsubsection{مقایسه عملکرد افراد در دور دوم بازی نسبت به دور اول در تمام گروه‌ها}

ابتدا توزیع عملکرد افراد در هر گروه را بررسی می‌کنیم. در شکل \ref{fig:p1sNorm} و \ref{fig:p2sNorm} و \ref{fig:p3sNorm} به ترتیب توزیع عملکرد افراد در گروه اصلی، گروه کنترل ۱ و گروه کنترل ۲ نمایش داده شده است.

\begin{figure}
\centering
\includegraphics[scale=0.5]{Figures/p1sNorm.png}
\caption{\label{fig:p1sNorm}
نمودار احتمال نرمال بودن و هیستوگرام عملکرد گروه اصلی در دور اول و دوم بازی
}
\end{figure}

\begin{figure}
\centering
\includegraphics[scale=0.5]{Figures/p2sNorm.png}
\caption{\label{fig:p2sNorm}
نمودار احتمال نرمال بودن و هیستوگرام عملکرد گروه کنترل ۱ در دور اول و دوم بازی
}
\end{figure}

\begin{figure}
\centering
\includegraphics[scale=0.5]{Figures/p3sNorm.png}
\caption{\label{fig:p3sNorm}
نمودار احتمال نرمال بودن و هیستوگرام عملکرد گروه کنترل ۲ در دور اول و دوم بازی
}
\end{figure}

در شکل‌ها مشاهده می‌شود که توزیع‌ها به شکلی نیست که بتوانیم از روش t استفاده کنیم به همین دلیل در این قسمت از روش ویلکاکسون استفاده خواهیم کرد. نتایج به دست آمده از این روش در جدول \ref{tbl:scoreCompWilcox} نمایش داده شده است. با توجه به نتایج نمایش داده شده در جدول واضح است که در هیچ کدام از گروه‌ها عملکرد در دور دوم بازی تغییر معناداری نسبت به عملکرد در دور اول بازی نداشته است.


\begin{table}[]
\begin{scriptsize}
\begin{center}
\centering
\caption{مقایسه آماری عملکرد افراد در دور دوم بازی نسبت به دور اول}
\label{tbl:scoreCompWilcox}
\renewcommand{\arraystretch}{2}
\begin{tabular}{|c|c|c|c|}
\hline
\textbf{آزمایش}& \textbf{میانه عملکرد دور اول (درصد)} & \textbf{میانه عملکرد دور دوم (درصد)} & p-value\\ \hline
\textbf{\begin{tabular}[c]{@{}c@{}}گروه اصلی\end{tabular}} & ۳۴٪& ۳۳٪& ۰/۹۹ \\ \hline
\textbf{\begin{tabular}[c]{@{}c@{}}گروه کنترل ۱\end{tabular}} & ۳۳٪ & ۳۷٪& ۰/۲۳\\ \hline
\textbf{\begin{tabular}[c]{@{}c@{}}گروه کنترل ۲\end{tabular}} &۴۰/۲٪ & ۴۰/۶ & ۰/۷۷\\ \hline
\end{tabular}
\end{center}
\end{scriptsize}
\end{table}

موضوعی که اینجا مطرح می‌شود این است که افرادی که در دور اول بازی کردن امتیاز بالایی را کسب کرده‌اند امکان زیادی برای پیشرفت در دور دوم ندارند اما افرادی که در دور اول عملکردشان ضعیف بوده است می‌توانند پیشرفت بیشتری در دور دوم داشته باشند. به همین دلیل علاوه بر بررسی داده‌های اصلی داده‌های افرادی را که در دور اول عملکرد ضعیف‌تری داشتند را مجددا بررسی کردیم. برای این کار از هر گروه ۱۲ نفری را که عملکرد ضعیف‌تری در دور اول بازی داشتند انتخاب کردیم. نتایج به دست آمده در جدول \ref{tbl:scoreCompWilcoxWeakOnes} نمایش داده شده‌اند. با توجه به جدول مشاهده می‌شود در گروه کنترل ۱ بهبود معناداری در قسمت ۲ نسبت به قسمت ۱ مشاهده شده است ولی در گروه اصلی و گروه کنترل ۱ بهبود معناداری مشاهده نمی‌شود.

\begin{table}[]
\begin{scriptsize}
\begin{center}
\centering
\caption{مقایسه آماری عملکرد افراد ضعیف در دور دوم بازی نسبت به دور اول}
\label{tbl:scoreCompWilcoxWeakOnes}
\renewcommand{\arraystretch}{2}
\begin{tabular}{|c|c|c|c|}
\hline
\textbf{آزمایش}& \textbf{میانه عملکرد دور اول (درصد)} & \textbf{میانه عملکرد دور دوم (درصد)} & p-value\\ \hline
\textbf{\begin{tabular}[c]{@{}c@{}}گروه اصلی\end{tabular}} & ۳۰/۶٪& ۲۹/۸٪& ۰/۹۳ \\ \hline
\textbf{\begin{tabular}[c]{@{}c@{}}گروه کنترل ۱\end{tabular}} & ۲۷/۴٪ & ۳۷/۲٪& ۰/۰۱۹\\ \hline
\textbf{\begin{tabular}[c]{@{}c@{}}گروه کنترل ۲\end{tabular}} &۳۳/۸٪ & ۴۰/۶ & ۰/۳۵\\ \hline
\end{tabular}
\end{center}
\end{scriptsize}
\end{table}

\subsubsection{مقایسه‌ی تغییرات عملکرد گروه اصلی، گروه کنترل ۱ و گروه کنترل ۲}

در این بخش می‌خواهیم تغییرات عملکرد گروه اصلی، گروه کنترل ۱ و گروه کنترل ۲ را با هم مقایسه کنیم. در شکل \ref{fig:impv123NormalPlot} \mls{نمودار احتمال نرمال بودن} و \mls{هیستوگرام} برای این سه دسته از داده‌ها رسم شده است. با توجه به شکل مشاهده می‌شود که توزیع داده‌ها برای انجام دادن آزمون t مناسب است.

\begin{figure}
\centering
\includegraphics[scale=0.5]{Figures/impv123NormalPlot.png}
\caption{\label{fig:impv123NormalPlot}
نمودار احتمال نرمال بودن و هیستوگرام تغییرات عملکرد گروه اصلی و گروه کنترل ۱ و گروه کنترل ۲
}
\end{figure}

نتیجه‌ی به دست آمده از آزمون t در جدول \ref{tbl:impv123ttest} نمایش داده شده است. همانطور که در جدول مشاهده می‌شود مقدار p-value در تمامی موارد بیشتر از سطح معناداری است و بنابراین نمی‌توانیم نتیجه بگیریم میانگین این گروه‌ها تفاوت معناداری با یکدیگر دارند.

\begin{table}[]
\begin{scriptsize}
\begin{center}
\centering
\caption{نتایج آزمون آماری t ، مقایسه‌ی میانگین تغییرات عملکرد گروه اصلی، گروه کنترل ۱ و گروه کنترل ۲}
\label{tbl:impv123ttest}
\renewcommand{\arraystretch}{2}
\begin{tabular}{|c|c|}
\hline
\textbf{میانگین تغییرات عملکرد گروه اصلی}& ۱/۷۹٪\\ \hline
\textbf{\begin{tabular}[c]{@{}c@{}}میانگین تغییرات عملکرد گروه کنترل ۱\end{tabular}} & ۹/۳۷٪ \\ \hline
\textbf{\begin{tabular}[c]{@{}c@{}}میانگین تغییرات عملکرد گروه کنترل ۲\end{tabular}} & ۶/۹۷٪ \\ \hline
\textbf{\begin{tabular}[c]{@{}c@{}}مقدار p-value در مقایسه گروه اصلی و گروه کنترل ۱\end{tabular}} & ۰/۱۶ \\ \hline
\textbf{\begin{tabular}[c]{@{}c@{}}مقدار p-value در مقایسه گروه اصلی و گروه کنترل ۲\end{tabular}} & ۰/۲۲ \\ \hline
\textbf{\begin{tabular}[c]{@{}c@{}}مقدار p-value در مقایسه گروه کنترل ۱ و گروه کنترل ۲\end{tabular}} & ۰/۳۸ \\ \hline
\end{tabular}
\end{center}
\end{scriptsize}
\end{table}

مشابه بخش قبل با توجه به اینکه افرادی که در ابتدا عملکرد ضعیف‌تری داشته‌اند احتمال بهبود بیشتری دارند مجددا افراد ضعیف را جدا کرده و میانگین تغییرات عملکرد آنها را بررسی می‌کنیم. نتیجه‌ی این بررسی در جدول \ref{tbl:impv123ttestWeakOnes} نمایش داده شده است.

\begin{table}[]
\begin{scriptsize}
\begin{center}
\centering
\caption{نتایج آزمون آماری t ، مقایسه‌ی میانگین تغییرات عملکرد افراد ضعیف‌تر در گروه اصلی، گروه کنترل ۱ و گروه کنترل ۲}
\label{tbl:impv123ttestWeakOnes}
\renewcommand{\arraystretch}{2}
\begin{tabular}{|c|c|}
\hline
\textbf{میانگین تغییرات عملکرد گروه اصلی}& ۱/۴۴٪\\ \hline
\textbf{\begin{tabular}[c]{@{}c@{}}میانگین تغییرات عملکرد گروه کنترل ۱\end{tabular}} & ۱۸/۲۹٪ \\ \hline
\textbf{\begin{tabular}[c]{@{}c@{}}میانگین تغییرات عملکرد گروه کنترل ۲\end{tabular}} & ۷/۲۸٪ \\ \hline
\textbf{\begin{tabular}[c]{@{}c@{}}مقدار p-value در مقایسه‌ی گروه اصلی و گروه کنترل ۱\end{tabular}} & ۰/۰۳ \\ \hline
\textbf{\begin{tabular}[c]{@{}c@{}}مقدار p-value در مقایسه‌ی گروه اصلی و گروه کنترل ۲\end{tabular}} & ۰/۲۶ \\ \hline
\textbf{\begin{tabular}[c]{@{}c@{}}مقدار p-valueدر مقایسه‌ی گروه کنترل ۱ و گروه کنترل ۲\end{tabular}} & ۰/۱۴ \\ \hline
\end{tabular}
\end{center}
\end{scriptsize}
\end{table}

همانطور که مشخص است مقدار p-value در مقایسه‌ی گروه اصلی و گروه کنترل ۱ کمتر از سطح معناداری است و این به این معنا است که میانگین تغییرات عملکرد افراد ضعیف در گروه کنترل ۱ به طرز معناداری بیشتر از میانگین تغییرات عملکرد افراد ضعیف در گروه اصلی است ولی در سایر موارد تفاوت معناداری مشاهده نمی‌شود.


\subsubsection{مقایسه زمان پاسخگویی دور دوم نسبت به دور اول بازی در تمام گروه‌ها}

ابتدا باید بررسی کنیم که توزیع داده‌های مربوط به زمان به شکلی هست که بتوانیم از آزمون t استفاده کنیم یا خیر. به این منظور نمودار احتمال نرمال بودن و هیستوگرام هر بخش از داده‌ها را بررسی می‌کنیم. در شکل‌های \ref{fig:p1tNorm} و \ref{fig:p2tNorm} و \ref{fig:p3tNorm} نمودار احتمال نرمال بودن و هیستوگرام گروه اصلی، گروه کنترل ۱ و گروه کنترل ۲ به ترتیب نمایش داده شده است.

\begin{figure}
\centering
\includegraphics[scale=0.5]{Figures/p1tNorm.png}
\caption{\label{fig:p1tNorm}
نمودار احتمال نرمال بودن و هیستوگرام میانگین زمان پاسخگویی گروه اصلی
}
\end{figure}

\begin{figure}
\centering
\includegraphics[scale=0.5]{Figures/p2tNorm.png}
\caption{\label{fig:p2tNorm}
نمودار احتمال نرمال بودن و هیستوگرام میانگین زمان پاسخگویی گروه کنترل ۱
}
\end{figure}

\begin{figure}
\centering
\includegraphics[scale=0.5]{Figures/p3tNorm.png}
\caption{\label{fig:p3tNorm}
نمودار احتمال نرمال بودن و هیستوگرام میانگین زمان پاسخگویی گروه کنترل ۲
}
\end{figure}

با توجه به نتایج به دست آمده در برخی از گروه‌ها توزیع داده‌ها به شکلی نیست که بتوانیم از آزمون t استفاده کنیم. بنابراین در این بخش از روش ویلکاکسون استفاده خواهیم کرد.
ابتدا زمان پاسخگویی تمام گروه‌ها در دور دوم بازی را با دور اول مقایسه می‌کنیم. نتایج این مقایسه در جدول \ref{tbl:ResponseTimeComp} نمایش داده شده است. همانطور که مشخص است با توجه به مقدار p-value میانه زمان پاسخگویی در گروه کنترل ۱ در دور دوم بازی به شکل معناداری نسبت به دور اول بازی کاهش پیدا کرده است ولی در گروه اصلی و گروه کنترل ۲ تفاوت معناداری مشاهده نمی‌کنیم.
\begin{table}[]
\begin{scriptsize}
\begin{center}
\centering
\caption{مقایسه آماری زمان پاسخگویی دور دوم بازی نسبت به دور اول}
\label{tbl:ResponseTimeComp}
\renewcommand{\arraystretch}{2}
\begin{tabular}{|c|c|c|c|}
\hline
\textbf{آزمایش}& \textbf{میانه زمان پاسخگویی دور اول (میلی ثانیه)} & \textbf{میانه زمان پاسخگویی دور دوم (میلی ثانیه)} & p-value\\ \hline
\textbf{\begin{tabular}[c]{@{}c@{}}گروه اصلی\end{tabular}} & ۹۴/۸& ۸۷/۵۵& ۰/۰۵۴ \\ \hline
\textbf{\begin{tabular}[c]{@{}c@{}}گروه کنترل ۱\end{tabular}} & ۹۸/۷ & ۸۳/۱& ۰/۰۰۰۲\\ \hline
\textbf{\begin{tabular}[c]{@{}c@{}}گروه کنترل ۲\end{tabular}} &۹۹/۲۵ & ۸۳/۶۵ & ۰/۰۶\\ \hline
\end{tabular}
\end{center}
\end{scriptsize}
\end{table}

\subsubsection{مقایسه تغییرات زمان پاسخگویی گروه اصلی، گروه کنترل ۱ و گروه کنترل ۲ نسبت به یکدیگر}
مانند بخش‌های قبل ابتدا بررسی می‌کنیم که توزیع داده‌ها مناسب استفاده از آزمون t هستند یا خیر. نمودار احتمال نرمال بودن و هیستوگرام مربوط به تغییرات زمان پاسخگویی هر سه بخش در شکل \ref{fig:dtNorm} نمایش داده شده است.

\begin{figure}
\centering
\includegraphics[scale=0.5]{Figures/dtNorm.png}
\caption{\label{fig:dtNorm}
نمودار احتمال نرمال بودن و هیستوگرام تغییرات زمان پاسخگویی هر سه گروه
}
\end{figure}

همانطور که از شکل مشخص است توزیع داده‌ها در برخی موارد به شکلی نیست که بتوانیم از آزمون t استفاده کنیم. بنابراین در این بخش از روش ویلکاکسون استفاده خواهیم کرد. نتایج در جدول \ref{tbl:dtCompWilcox} نمایش داده شده است. همانطور که از موارد نمایش داده شده در جدول مشخص است تغییرات زمان پاسخگویی گروه کنترل ۱ به طرز معناداری از تغییرات زمان پاسخگویی گروه اصلی کمتر است اما سایر موارد تفاوت معناداری با یکدیگر ندارند.

\begin{table}[]
\begin{scriptsize}
\begin{center}
\centering
\caption{نتایج آزمون ویلکاکسون، مقایسه تغییرات زمان پاسخگویی در سه گروه }
\label{tbl:dtCompWilcox}
\renewcommand{\arraystretch}{2}
\begin{tabular}{|c|c|}
\hline
\textbf{\begin{tabular}[c]{@{}c@{}}میانه تغییرات زمان پاسخگویی گروه اصلی\end{tabular}} & ۹/۸۲٪- \\ \hline
\textbf{\begin{tabular}[c]{@{}c@{}}میانه تغییرات زمان پاسخگویی گروه کنترل ۱\end{tabular}} & ۱۸/۰۱٪- \\ \hline
\textbf{\begin{tabular}[c]{@{}c@{}}میانگین تغییرات زمان پاسخگویی گروه کنترل ۲\end{tabular}} & ۱۳/۲۳٪- \\ \hline
\textbf{\begin{tabular}[c]{@{}c@{}}مقدار p-value در مقایسه‌ی گروه اصلی و گروه کنترل ۱\end{tabular}} & ۰/۰۱ \\ \hline
\textbf{\begin{tabular}[c]{@{}c@{}}مقدار p-value در مقایسه‌ی گروه اصلی و گروه کنترل ۲\end{tabular}} & ۰/۸۱ \\ \hline
\textbf{\begin{tabular}[c]{@{}c@{}}مقدار p-valueدر مقایسه‌ی گروه کنترل ۱ و گروه کنترل ۲\end{tabular}} & ۰/۰۹ \\ \hline
\end{tabular}
\end{center}
\end{scriptsize}
\end{table}

\chapter{جمع‌بندی و نکته‌های پایانی}
\label{chapter:conclusion}
\thispagestyle{plain}



\newpage
\thispagestyle{empty}
\mbox{}

%  : \linespread{1.2}
\linespread{1}


\small{
\bibliographystyle{ieeetr-fa}
\renewcommand{\bibname}{مراجع}
\clearpage
\bibliography{ref}
\addcontentsline{toc}{chapter}{مراجع}
}

\newpage
\thispagestyle{empty}
\mbox{}
\begin{multicols}{2}
\begin{doublespace}

\glossarystyle{mylistLa}
\printglossary[type=latin]
\addcontentsline{toc}{chapter}{واژه‌نامه انگلیسی به فارسی}

\newpage
\thispagestyle{empty}
\mbox{}

\clearpage
\glossarystyle{mylistFa}
\printglossary[type=persian]
\addcontentsline{toc}{chapter}{واژه‌نامه فارسی به انگلیسی}
\end{doublespace}
\end{multicols}

\begin{latin}
\pagestyle{empty}


% LATIN ABSTRACT
\newpage
\thispagestyle{empty}
\mbox{}
\newpage
\vspace*{-2cm}
{\centering\small{\bf{Abstract}} \par \vskip .1cm}
\begin{doublespace} \normalsize

In recent years, due to the increasing amount of data available on the internet, the use of search engines to retrieve relevant information from the World Wide Web has become pervasive. Among the huge number of websites, the ones which succeed to appear more frequently and in higher ranks of search engine results would receive more visitors. So, spammers struggle to achieve a higher than deserved rank for their websites using some illegal techniques called web spamming. 
Although various methods have been used for combatting web spamming, we could basically categorize them into three groups: content-based methods, link-based methods, and the methods based on miscellaneous data. 
In this thesis, we focus on content-based and link-based methods, and also their combination.

Despite the existence of many spam detection methods, the search engines do not perform well in detecting Persian spam websites. Thus, in this thesis, after preparing a corpus of spam and non-spam Persian websites, we analyze the effectiveness of many previously proposed content-based features on detecting Persian spam websites. To improve the performance of classification, we present a number of new content-based features and examine a number of feature selection method. As another approach, we propose a new Persian spam detection system which uses an improved version of bag-of-words model and has better performance in detecting Persian web spam. Due to the prevalence of link-based spamming methods, we analyze some of these methods and propose two new algorithms which do not have the weaknesses of previous methods. In the first algorithm, to improve the process of label propagation, we use three mechanisms: optimized seed selection, edge weighting, and seed expansion. In the second algorithm, we improve the quality of websites ranking, using label propagation in both forward and backward directions. Finally, we propose a combined method, which uses the content-based probability of being spam (non-spam) to propagate the spam (non-spam) score of websites. Using this method, we increase the performance of ranking websites.
 
Finally, to evaluate the proposed methods and compare their performance with the existing methods for this task, we have conducted several experiments on different datasets. Experiment results indicate that the proposed methods have a good performance in detecting web spam.




\end{doublespace}
{\par\vspace{2mm}}
\noindent\textbf{Keywords: }\textit{Spamming, Web Spam, Spam Detection, Label Propagation, Content-Based Features}
% END OF LATIN ABSTRACT

\newpage
\mbox{}


%%%%%%%%%%%%%%%%%%%%%%%%%%%%%
% LATIN TITLE PAGE
%%%%%%%%%%%%%%%%%%%%%%%%%%%%%
\font\titlefont=cmssbx10 scaled 2074
\font\supervisorfont=cmbxti10
\newpage
\thispagestyle{empty}
\begin{center}
\begin{tabular}{lp{7cm}r}
\includegraphics[width=3.8cm]{Figures/eng.png} & & \includegraphics[width=2.8cm]{Figures/ut.png} \\
\end{tabular}

\vskip 1cm
\Large{\bfseries
University of Tehran \par
School of Electrical and Computer Engineering}
\large
\par
\vskip 1.5cm
\addtolength{\baselineskip}{5mm}
{\titlefont User Experience Evaluation in Attention Cognitive Game and its Usage for Improving Beginners} \par
\addtolength{\baselineskip}{-5mm}
\vskip 1cm
{\bfseries By}\par
{\Large\bfseries Elaheh Abolhassani Shahreza}\par
\vskip 1cm
Supervisors: \\
{\supervisorfont\Large Dr. Majid Nili} \\
{\supervisorfont\Large Dr. Hadi Moradi} \\
%{\supervisorfont\Large Dr. Masoud Asadpour}%
\par
\vskip 2cm
{A thesis submitted to the Graduate Studies Office \\ in partial fulfillment of the requirements \\ for the degree of Master of Science \par
in
\par
\vskip .7cm
\large Computer Engineering}
\par
\vskip .04cm
{September 2017}
\par
\vfill
\end{center}
%%%%%%%%%%%%%%%%%%%%%%%%%%%%%
% END OF LATIN TITLE

\end{latin}
\end{document}‍
